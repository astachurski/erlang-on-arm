\chapter{Lista instrukcji maszyny wirtualnej BEAM}
\label{cha:operacjeBeam}
%---------------------------------------------------------------------------

Dodatek zawiera listę instrukcji maszyny wirtualnej BEAM, jakie może zawierać skompilowany kod pośredni przez nią wykonywany oraz sposób zapisu argumentów dla instrukcji.

Kod danej operacji zajmuje zawsze 1 bajt w pliku ze skompilowanym kodem pośrednim modułu. 
Argumenty mogą zajmować więcej przestrzeni, zgodnie z opisem w sekcji \ref{sec:opsTypes}.

Kolejość bajtów w zapisie kodu pośredniego to zawsze \emph{big endian}.


\section{Typy argumentów}
\label{sec:opsTypes}
%---------------------------------------------------------------------------
Argumentem jest zawsze liczba całkowita, reprezentująca taką wartość liczbową albo indeks w odpowiedniej tablicy z wartościami (pierwszym indeksem takiej tablicy jest 0). W związku z tym argumenty mogą być różnego typu. Aby rozróżnić argument jednego typu od drugiego poddaje się je odpowiedniemu tagowaniu. Operację wykonuje się bezpośrednio na argumencie, jeśli jest dostatecznie mały, lub na odpowiednim nagłówku poprzedzającym argument. Rozróżnienie to jest spowodowane oszczędnością rozmiaru kodu pośredniego, który musi być przechowywany w pamięci.

Każdy z tagów, które zostały wymienione w tabeli \ref{table:codeTags}, jest możliwy do zapisania przy użyciu 3 bitów, które zajmują najmniej znaczące bity argumentu.
Jednak w kodowaniu binarnym do zapisu typu używane są dodatkowo 1 lub 2 bity. Dzięki nim możliwe jest rozróżnienie pomiędzy argumentami zapisanymi przy użyciu różnej liczby bajtów.

Tagowanie odbywa się za pomocą następującej operacji:

$${(0000\text{XXXX} \ll N)}_{(2)} \oplus {000\text{\textbf{SS}}\text{\textbf{TTT}}}_{(2)},$$
gdzie $\text{XXXX}_{(2)}$ jest tagowaną liczbą, 
$N=4 \text{ lub } 5$, 
$\text{\textbf{SS}}_{(2)}$ są dodatkowymi bitami znakującymi rozmiar argumentu, a
$\text{\textbf{TTT}}_{(2)}$ jest danym tagiem.

\begin{longtable}{|c|c|p{9cm}|}
\hline

\multicolumn{2}{|c|}{\textbf{Tag}} & \multirow{2}{*}{\textbf{Typ}} \\
\cline{1-2}
\textbf{binarnie} & \textbf{dziesiętnie} & \\
\hline
\endfirsthead

000 & 0 & uniwersalny indeks, np. do tablicy stałych \\
\hline
001 & 1 & liczba całkowita \\ 
\hline
010 & 2 & indeks do tablicy atomów \\
\hline
011 & 3 & numer rejestru X maszyny wirtualnej \\
\hline
100 & 4 & numer rejestru Y maszyny wirtualnej \\
\hline
101 & 5 & etykieta, używana w funkcjach skoku \\
\hline
111 & 7 & złożone wyrażenie (np. lista, liczba zmiennoprzecinkowa) \\
\hline

\caption{Tagi typów danych w pliku ze skompilowanym modułem} 
\label{table:codeTags} \\
\end{longtable}

Jeżeli tagowana liczba jest nieujemna, mniejsza od 16 (możliwe jest zapisanie jej przy użyciu 4 bitów) to argument jest zapisany przy użyciu jednego bajtu a jego postać binarna to:
$$ \text{X}_1\text{X}_2\text{X}_3\text{X}_4\mathbf{0}\text{\textbf{TTT}}_{(2)}, $$
gdzie ${\text{X}_1\text{X}_2\text{X}_3\text{X}_4}_{(2)}$ to tagowana liczba, $\text{X}_1$ jest jej najbardziej znaczącym bitem, a $\text{TTT}_{(2)}$ to tag danego typu argumentu.

Na przykład, atom, który w tablicy atomów modułu ma indeks $2_{10} = 10_{2}$, po zakodowaniu będzie miał postać:
$$0010\mathbf{0010}_{2} = 22_{16} = 34_{10}.$$

W przypadku, gdy liczba jest nieujemna, mniejsza lub równa 16, a mniejsza od 2048 (możliwe jest jej zapisanie przy użyciu 11 bitów), argument jest zapisany przy użyciu dwóch bajtów, których postać binarna to:
$$  {\text{X}_1\text{X}_2\text{X}_3\mathbf{01}\text{\textbf{TTT}} \enskip \text{X}_4\text{X}_5\text{X}_6\text{X}_7\text{X}_8\text{X}_9\text{X}_{10}\text{X}_{11}}_{(2)}, $$
gdzie ${\text{X}_1 ... \text{X}_{11}}_{(2)}$ to tagowana liczba, $\text{X}_1$ jest jej najbardziej znaczącym bitem, a ${\text{TTT}}_{(2)}$ to tag danego typu argumentu.

Na przykład, liczba całkowita $565_{10} = {010 \enskip 00110101}_{2}$ po zakodowaniu będzie miała postać:
$${010\mathbf{01001} \enskip 00110101}_{2} = 4935_{16} = 18741_{10}.$$

Jeżeli argument jest liczbą ujemną lub dodatnią wymagającą w zapisie dwójkowym więcej niż 11 bitów to liczba taka zapisywana jest binarnie w kodzie uzupełnień do dwóch (U2) poprzedzona odpowiednim nagłówkiem.

Jeżeli zakodowaną liczbę można zapisać na nie więcej niż 8 bajtach, to nagłówek ma następującą postać:

$$ {\text{N}_1\text{N}_2\text{N}_3 \mathbf{11} \text{\textbf{TTT}}}_{(2)}, $$
gdzie ${\text{N}_1\text{N}_2\text{N}_3}_{(2)}$ to rozmiar argumentu w bajtach pomniejszony o 2 (jeżeli argument jest liczbą ujemną zajmującą 1 bajt to powinien on zostać dopełniony do 2 bajtów), $\text{N}_1$ jest jego najbardziej znaczącym bitem, a $\text{TTT}_{(2)}$ to tag danego typu argumentu.

Na przykład, aby zapisać na dwóch bajtach liczbę $-21_{10} = {11111111 \enskip 11101011}_{U2}$, jej postać binarną należy poprzedzić nagłówkiem:

$${000\mathbf{11001}}_{2} = 19_{16} = 25_{10}.$$

Jeżeli do zapisania liczby w kodzie uzupełnień do dwóch potrzeba przynajmniej 9 bajtów, wtedy nagłówek jest dwubajtowy i ma postać:

$$ {11111\text{\textbf{TTT}} \enskip \text{N}_1\text{N}_2\text{N}_3\text{N}_4 \mathbf{0000}}_{(2)}, $$
gdzie ${\text{N}_1\text{N}_2\text{N}_3\text{N}_4}_{(2)}$ to rozmiar argumentu w bajtach pomniejszony o 9, $\text{N}_1$ jest jego najbardziej znaczącym bitem, a $\text{TTT}_{(2)}$ to tag danego typu argumentu.

Na przykład, w celu zapisania liczby $2^{(15 \times 8)-1}-1$ na 15 bajtach, należy zapis tej liczby w kodzie U2 poprzedzić następującym nagłówkiem:

$${11111\mathbf{001} \enskip 0110\mathbf{0000}}_{2} = \text{F}960_{16} = 63840_{10}.$$

\section{Lista instrukcji}
\label{sec:opsOps}
%---------------------------------------------------------------------------
W tabeli \ref{tab:ops} zawarto listę instrukcji rozumianych przez maszynę wirtualną BEAM wraz z jednobajtowym kodem operacji, listą jej argumentów i krótkim opisem działania.

Instrukcje nieużywane przez kompilator Erlanga w wersji R16 zostały pominięte.

\begin{longtable}{|c|c|p{5cm}|p{7cm}|}
\hline

\multicolumn{2}{|c|}{\textbf{Kod operacji}} & \multirow{2}{*}{\textbf{Nazwa operacji i jej argumenty}} & \multirow{2}{*}{\textbf{Opis operacji i uwagi}} \\
\cline{1-2}
\textbf{hex} & \textbf{dec} & & \\
\hline
\endfirsthead

01 & 1 & \texttt{label Lbl} & Wprowadza lokalną dla danego modułu etykietę identyfikującą aktualne miejsce w kodzie. \\
\hline
02 & 2 & \texttt{func\_info M F A} & Definiuje funkcję \texttt{F}, w module \texttt{M} o arności \texttt{A}. \\
\hline
03 & 3 & \texttt{int\_code\_end} & Oznacza koniec kodu.  \\
\hline
04 & 4 & \texttt{call A Lbl} & Wywołuje funkcję o arności \texttt{A} znajdującą się pod etykietą \texttt{Lbl}. Zapisuje następną instrukcję jako adres powrotu (rejestr \textbf{CP}). \\
\hline
05 & 5 & \texttt{call\_last Arity Lbl Dest} & Wywołuje prawostronnie rekursywnie funkcję o arności \texttt{Arity} znajdującą się pod etykietą \texttt{Lbl}. Nie zapisuje adresu powrotu. Przed wywołaniem zwalnia \texttt{Dest} słów pamięci na stosie.\\
\hline
06 & 6 & \texttt{call\_only Arity Lbl} & Wywołuje prawostronnie rekursywnie funkcję o arności \texttt{Arity} znajdującą się pod etykietą \texttt{Lbl}. Nie zapisuje adresu powrotu. \\
\hline
07 & 7 & \texttt{call\_ext Arity Dest} & Wywołuje zewnętrzną funkcję o arności \texttt{Arity} mającą indeks \texttt{Dest} w tablicy funkcji zewnętrznych. Zapisuje następną instrukcję jako adres powrotu (wskaźnik \textbf{CP}). \\
\hline
08 & 8 & \texttt{call\_ext\_last Arity Des Dea} & Wywołuje prawostronnie rekursywnie zewnętrzną funkcję o arności \texttt{Arity} mającą indeks \texttt{Des} w tablicy funkcji zewnętrznych. Nie zapisuje adresu powrotu. Przed wywołaniem zwalnia \texttt{Dea} słów pamięci na stosie.\\
\hline
09 & 9 & \texttt{bif0 Bif Reg} & Wywołuje wbudowaną funkcję \texttt{Bif/0}. Wynik zapisywany jest w rejestrze \texttt{Reg}. \\
\hline
0A & 10 & \texttt{bif1 Bif Arg Reg} & Wywołuje wbudowaną funkcję \texttt{Bif/1} z argumentem \texttt{Arg}. Wynik zapisywany jest w rejestrze \texttt{Reg}. \\
\hline
0B & 11 & \texttt{bif2 Bif Arg1 Arg2 Reg} & Wywołuje wbudowaną funkcję \texttt{Bif/2} z argumentami \texttt{Arg1}, \texttt{Arg2}. Wynik zapisywany jest w rejestrze \texttt{Reg}. \\
\hline
0C & 12 & \texttt{allocate SN L} & Alokuje miejsce dla \texttt{SN} słów na stosie. Używanych jest \texttt{L} rejestrów \textbf{X}, gdyby w trakcie alokacji konieczne było uruchomienie \emph{garbage collectora}. Zapisuje \textbf{CP} na stosie. \\
\hline
0D & 13 & \texttt{allocate\_heap SN HN L} & Alokuje miejsce dla \texttt{SN} słów na stosie. Upewnia się że na stercie jest \texttt{HN} wolnych słów. Używanych jest \texttt{L} rejestrów \textbf{X}, gdyby w trakcie alokacji konieczne było uruchomienie \emph{garbage collectora}. Zapisuje \textbf{CP} na stosie. \\
\hline
0E & 14 & \texttt{allocate\_zero SN L} & Tak jak \texttt{allocate/2}, ale zaalokowana pamięć jest wyzerowana. \\
\hline
0F & 15 & \texttt{allocate\_heap\_zero SN HN L} & Tak jak \texttt{allocate\_heap/3}, ale zaalokowana pamięć jest wyzerowana. \\
\hline
10 & 16 & \texttt{test\_heap HN L} & Upewnia się że na stercie jest \texttt{HN} wolnych słów. Używanych jest \texttt{L} rejestrów \textbf{X}, gdyby w trakcie konieczne było uruchomienie \emph{garbage collectora}.\\
\hline
11 & 17 & \texttt{init N} & Zeruje \texttt{N}-te słowo na stosie.\\
\hline
12 & 18 & \texttt{deallocate N} & Przywraca \textbf{CP} ze stosu i dealokuje \texttt{N}+1 słów ze stosu.\\
\hline
13 & 19 & \texttt{return} & Wraca do adresu zapisanego we wskaźniku \textbf{CP}.\\
\hline
14 & 20 & \texttt{send} & Wysyła wiadomość z rejestru \textbf{X1} do procesu w rejestrze \textbf{X0}. \\
\hline
15 & 21 & \texttt{remove\_message} & Usuwa aktualną wiadomość z kolejki wiadomości. Zapisuje wskaźnik do niej w rejestrze \textbf{X0}. Usuwa aktywne przeterminowanie (\emph{timeout}). \\
\hline
16 & 22 & \texttt{timeout} & Resetuje wskaźnik \textbf{SAVE}. Czyści flagę przeterminowanie. \\
\hline
17 & 23 & \texttt{loop\_rec Lbl Src} & Zapisuje kolejną wiadomość w kolejce wiadomości \textbf{Src} w rejestrze \textbf{R0}. Jeśli jest pusta wykonuje skok do etykiety \textbf{Lbl}. \\
\hline
18 & 24 & \texttt{loop\_rec\_end Lbl} & Ustawia wskaźnik \textbf{SAVE} na kolejną wiadomość w kolejce wiadomości i wykonuje skok do etykiety \textbf{Lbl}. \\
\hline
19 & 25 & \texttt{wait Lbl} & Zawiesza proces aż do otrzymania wiadomości, który zostanie wznowiony na początku bloku \texttt{receive} w etykiecie \texttt{Lbl}.\\
\hline
1A & 26 & \texttt{wait\_timeout Lbl T} & Zawiesza proces jak \texttt{wait}. Ustawia przeterminowanie \texttt{T} i zapisuje następną instrukcję, która zostanie wykonana jeśli przeterminowanie się zrealizuje.\\
\hline
27 & 39 & \texttt{is\_lt Lbl Arg1 Arg2} & Porównuje \texttt{Arg1} z \texttt{Arg2} i wykonuje skok do \texttt{Lbl} jeśli \texttt{Arg1} jest większe lub równe od \texttt{Arg2}.\\
\hline
28 & 40 & \texttt{is\_ge Lbl Arg1 Arg2} & Porównuje \texttt{Arg1} z \texttt{Arg2} i wykonuje skok do \texttt{Lbl} jeśli \texttt{Arg1} jest mniejsze \texttt{Arg2}.\\
\hline
29 & 41 & \texttt{is\_eq Lbl Arg1 Arg2} & Porównuje \texttt{Arg1} z \texttt{Arg2} i wykonuje skok do \texttt{Lbl} jeśli \texttt{Arg1} jest arytmetycznie różne od \texttt{Arg2}.\\
\hline
2A & 42 & \texttt{is\_ne Lbl Arg1 Arg2} & Porównuje \texttt{Arg1} z \texttt{Arg2} i wykonuje skok do \texttt{Lbl} jeśli \texttt{Arg1} jest arytmetycznie równe \texttt{Arg2}.\\
\hline
2B & 43 & \texttt{is\_eq\_exact Lbl Arg1 Arg2} & Porównuje \texttt{Arg1} z \texttt{Arg2} i wykonuje skok do \texttt{Lbl} jeśli \texttt{Arg1} jest różne \texttt{Arg2}.\\
\hline
2C & 44 & \texttt{is\_ne\_exact Lbl Arg1 Arg2} & Porównuje \texttt{Arg1} z \texttt{Arg2} i wykonuje skok do \texttt{Lbl} jeśli \texttt{Arg1} jest równe \texttt{Arg2}.\\
\hline
2D & 45 & \texttt{is\_integer Lbl Arg1} & Sprawdza typ \texttt{Arg1} i skacze do \texttt{Lbl} jeśli nie jest on liczbą całkowitą.\\
\hline
2E & 46 & \texttt{is\_float Lbl Arg1} & Sprawdza typ \texttt{Arg1} i skacze do \texttt{Lbl} jeśli nie jest on liczbą rzeczywistą.\\
\hline
2F & 47 & \texttt{is\_number Lbl Arg1} & Sprawdza typ \texttt{Arg1} i skacze do \texttt{Lbl} jeśli nie jest on liczbą.\\
\hline
30 & 48 & \texttt{is\_atom Lbl Arg1} & Sprawdza typ \texttt{Arg1} i skacze do \texttt{Lbl} jeśli nie jest on atomem.\\
\hline
31 & 49 & \texttt{is\_pid Lbl Arg1} & Sprawdza typ \texttt{Arg1} i skacze do \texttt{Lbl} jeśli nie jest on identyfikatorem procesu.\\
\hline
32 & 50 & \texttt{is\_reference Lbl Arg1} & Sprawdza typ \texttt{Arg1} i skacze do \texttt{Lbl} jeśli nie jest on referencją.\\
\hline
33 & 51 & \texttt{is\_port Lbl Arg1} & Sprawdza typ \texttt{Arg1} i skacze do \texttt{Lbl} jeśli nie jest on portem.\\
\hline
34 & 52 & \texttt{is\_nil Lbl Arg1} & Sprawdza typ \texttt{Arg1} i skacze do \texttt{Lbl} jeśli nie jest on zerem (\textbf{nil}).\\
\hline
35 & 53 & \texttt{is\_binary Lbl Arg1} & Sprawdza typ \texttt{Arg1} i skacze do \texttt{Lbl} jeśli nie jest on binarią.\\
\hline
37 & 55 & \texttt{is\_list Lbl Arg1} & Sprawdza typ \texttt{Arg1} i skacze do \texttt{Lbl} jeśli nie jest on ani listą ani zerem.\\
\hline
38 & 56 & \texttt{is\_nonempty\_list Lbl Arg1} & Sprawdza typ \texttt{Arg1} i skacze do \texttt{Lbl} jeśli nie jest on niepustą listą.\\
\hline
39 & 57 & \texttt{is\_tuple Lbl Arg1} & Sprawdza typ \texttt{Arg1} i skacze do \texttt{Lbl} jeśli nie jest on krotką.\\
\hline
3A & 58 & \texttt{test\_arity Lbl Arg1 Arity} & Sprawdza arność krotki \texttt{Arg1} i skacze do \texttt{Lbl} jeśli nie jest ona równa \texttt{Arity}.\\
\hline
3B & 59 & \texttt{select\_val Arg Lbl Dest} & Skacze do etykiety \texttt{Dest[Arg]}. Jeśli nie istnieje skacze do \texttt{Lbl}. \\
\hline
3C & 60 & \texttt{select\_tuple\_arity Tuple Lbl Dest} & Sprawdza arność krotki \texttt{Tuple} i skacze do etykiety \texttt{Dest[Arity]}. Jeśli nie istnieje skacze do \texttt{Lbl}. \\
\hline
3D & 61 & \texttt{jump Lbl} & Skacze do etykiety \texttt{Lbl}. \\
\hline
3E & 62 & \texttt{catch Dest Lbl} & Tworzy nowy blok \texttt{catch}. Zapisuje etykietę \texttt{Lbl} na \texttt{Dest} miejscu na stosie. \\
\hline
3F & 63 & \texttt{catch\_end Dest} & Kończy blok \texttt{catch}. Wymazuje etykietę na miejscu \texttt{Dest} na stosie. \\
\hline
40 & 64 & \texttt{move Src Dest} & Przenosi wartość z \texttt{Src} do rejestru \text{Dest}. \\
\hline
41 & 65 & \texttt{get\_list Src Hd Tail} & Umieść głowę listy \texttt{Src} w rejestrze \texttt{Hd} i jej ogon w rejestrze \texttt{Tail}. \\
42 & 66 & \texttt{get\_tuple\_element Src Elem Dest} & Umieść element \texttt{Elem} krotki \texttt{Src} w rejestrze \texttt{Dest}.\\
\hline
43 & 67 & \texttt{get\_tuple\_element Elem Tuple Pos} & Umieść element \texttt{Elem} w krotce \texttt{Tuple} na pozycji \texttt{Pos}.\\
\hline
45 & 69 & \texttt{put\_list Hd Tail Dest} & Tworzy listę \texttt{[Hd|Tail]} na szczycie sterty i umieszcza ją w rejestrze \texttt{Dest}.\\
\hline
46 & 70 & \texttt{put\_tuple Dest Arity} & Tworzy krotkę o arności \texttt{Arity} na szczycie sterty i umieszcza ją w rejestrze \texttt{Dest}.\\
\hline
47 & 71 & \texttt{put Arg} & Umieszcza \texttt{Arg} w lokalnym kontekście.\\
\hline
48 & 72 & \texttt{badmatch Arg} & Rzuca wyjątek \texttt{badmatch} z argumentem \texttt{Arg}.\\
\hline
49 & 73 & \texttt{if\_end} & Rzuca wyjątek \texttt{if\_clause}.\\
\hline
4A & 74 & \texttt{case\_end Arg} & Rzuca wyjątek \texttt{case\_clause} z argumentem \texttt{Arg}.\\
\hline
4B & 75 & \texttt{call\_fun Arity} & Woła obiekt funkcyjny o arności \texttt{Arity}. Zakłada, że argumenty znajdują się w rejestrach \textbf{X0}...\textbf{X(\texttt{Arity-1})}, a lambda w rejestrze \textbf{X(\texttt{Arity})}. Zapisuje następną instrukcję we wskaźniku \textbf{CP}.\\
\hline
4D & 77 & \texttt{is\_function Lbl Arg1} & Sprawdza typu argumentu \texttt{Arg1} i skacze do \texttt{Lbl} jeśli nie jest on funkcją.\\
\hline
4E & 78 & \texttt{call\_ext\_only Arity Lbl} & Wywołuje prawostronnie rekursywnie zewnętrzną funkcję o arności \texttt{Arity} mającą indeks \texttt{Lbl} w tablicy funkcji zewnętrznych. Nie zapisuje adresu powrotu.\\
\hline
59 & 89 & \texttt{bs\_put\_integer/5} & \\
\hline
5A & 90 & \texttt{bs\_put\_binary/5} & \\
\hline
5B & 91 & \texttt{bs\_put\_float/5} & \\
\hline
5C & 92 & \texttt{bs\_put\_string/2} & \\
\hline
5E & 94 & \texttt{fclearerror/0} & \\
\hline
5F & 95 & \texttt{fcheckerror/1} & \\
\hline
60 & 96 & \texttt{fmove/2} & \\
\hline
61 & 97 & \texttt{fconv/2} & \\
\hline
62 & 98 & \texttt{fadd/4} & \\
\hline
63 & 99 & \texttt{fsub/4} & \\
\hline
64 & 100 & \texttt{fmul/4} & \\
\hline
65 & 101 & \texttt{fdiv/4} & \\
\hline
66 & 102 & \texttt{fnegate/3} & \\
\hline
67 & 103 & \texttt{make\_fun2/1} & \\
\hline
68 & 104 & \texttt{try/2} & \\
\hline
69 & 105 & \texttt{try\_end/1} & \\
\hline
6A & 106 & \texttt{try\_case/1} & \\
\hline
6B & 107 & \texttt{try\_case\_end/1} & \\
\hline
6C & 108 & \texttt{raise/2} & \\
\hline
6D & 109 & \texttt{bs\_init2/6} & \\
\hline
6F & 111 & \texttt{bs\_add/5} & \\
\hline
70 & 112 & \texttt{apply/1} & \\
\hline
71 & 113 & \texttt{apply\_last/2} & \\
\hline
72 & 114 & \texttt{is\_boolean/2} & \\
\hline
73 & 115 & \texttt{is\_function2/3} & \\
\hline
74 & 116 & \texttt{bs\_start\_match2/5} & \\
\hline
75 & 117 & \texttt{bs\_get\_integer2/7} & \\
\hline
76 & 118 & \texttt{bs\_get\_float2/7} & \\
\hline
77 & 119 & \texttt{bs\_get\_binary2/7} & \\
\hline
78 & 120 & \texttt{bs\_skip\_bits2/5} & \\
\hline
79 & 121 & \texttt{bs\_test\_tail2/3} & \\
\hline
7A & 122 & \texttt{bs\_save2/2} & \\
\hline
7B & 123 & \texttt{bs\_restore2/2} & \\
\hline
7C & 124 & \texttt{gc\_bif1/5} & \\
\hline
7D & 125 & \texttt{gc\_bif2/6} & \\
\hline
81 & 129 & \texttt{is\_bitstr/2} & \\
\hline
82 & 130 & \texttt{bs\_context\_to\_binary/1} & \\
\hline
83 & 131 & \texttt{bs\_test\_unit/3} & \\
\hline
84 & 132 & \texttt{bs\_match\_string/4} & \\
\hline
85 & 133 & \texttt{bs\_init\_writable/0} & \\
\hline
86 & 134 & \texttt{bs\_append/8} & \\
\hline
87 & 135 & \texttt{bs\_private\_append/6} & \\
\hline
88 & 136 & \texttt{trim/2} & \\
\hline
89 & 137 & \texttt{bs\_init\_bits/6} & \\
\hline
8A & 138 & \texttt{bs\_get\_utf8/5} & \\
\hline
8B & 139 & \texttt{bs\_skip\_utf8/4} & \\
\hline
8C & 140 & \texttt{bs\_get\_utf16/5} & \\
\hline
8D & 141 & \texttt{bs\_skip\_utf16/4} & \\
\hline
8E & 142 & \texttt{bs\_get\_utf32/5} & \\
\hline
8F & 143 & \texttt{bs\_skip\_utf32/4} & \\
\hline
90 & 144 & \texttt{bs\_utf8\_size/3} & \\
\hline
91 & 145 & \texttt{bs\_put\_utf8/3} & \\
\hline
92 & 146 & \texttt{bs\_utf16\_size/3} & \\
\hline
93 & 147 & \texttt{bs\_put\_utf16/3} & \\
\hline
94 & 148 & \texttt{bs\_put\_utf32/3} & \\
\hline
95 & 149 & \texttt{on\_load/0} & \\
\hline
96 & 150 & \texttt{recv\_mark/1} & \\
\hline
97 & 151 & \texttt{recv\_set/1} & \\
\hline
98 & 152 & \texttt{gc\_bif3 Lbl Live Bif Arg1 Arg2 Arg3 Reg} & Wywołuje funkcję wbudowaną \texttt{Bif/3} z argumentami \texttt{Arg1}, \texttt{Arg2}, \texttt{Arg3}. Wynik zapisuje w rejestrze \texttt{Reg}. W przypadku niepowodzenia skacze do etykiety \texttt{Lbl}. Uruchamia \emph{garbage collector} jeśli jest to konieczne, zachowując \texttt{Live} rejestrów \textbf{X}. \\
\hline
99 & 153 & \texttt{line N} & Znakuje aktualne miejsce jako linia o indeksie \texttt{N} w tablicy linii.\\
\hline


\caption{Lista operacji maszyny wirtualnej BEAM} 
\label{tab:ops} \\
\end{longtable}