\chapter{Lista instrukcji maszyny wirtualnej BEAM}
\label{cha:operacjeBeam}
%---------------------------------------------------------------------------

Dodatek zawiera listę instrukcji maszyny wirtualnej BEAM, jakie może zawierać skompilowany kod pośredni przez nią wykonywany oraz sposób zapisu argumentów dla instrukcji.

Kod danej operacji zajmuje zawsze 1 bajt w pliku ze skompilowanym kodem pośrednim modułu. 
Argumenty mogą zajmować więcej przestrzeni, zgodnie z opisem w sekcji \ref{sec:opsTypes}.

Kolejość bajtów w zapisie kodu pośredniego to zawsze \emph{big endian}.


\section{Typy argumentów}
\label{sec:opsTypes}
%---------------------------------------------------------------------------
Argumentem jest zawsze liczba całkowita, reprezentująca wartość liczbową bądź indeks w odpowiedniej tablicy z wartościami (pierwszym indeksem takiej tablicy jest 0). W związku z tym argumenty mogą być różnego typu. Aby rozróżnić argument jednego typu od drugiego poddaje się je odpowiedniemu tagowaniu. Operację wykonuje się bezpośrednio na argumencie, jeśli jest dostatecznie mały, lub na odpowiednim nagłówku poprzedzającym argument. Rozróżnienie to jest spowodowane oszczędnością rozmiaru kodu pośredniego, który musi być przechowywany w pamięci.

Każdy z tagów, które zostały wymienione w tabeli \ref{table:codeTags}, jest możliwy do zapisania przy użyciu 3~bitów, które zajmują najmniej znaczące bity argumentu.
Jednak w kodowaniu binarnym do zapisu typu używane są dodatkowo 1 lub 2 bity. Dzięki nim możliwe jest rozróżnienie pomiędzy argumentami zapisanymi przy użyciu różnej liczby bajtów.

Tagowanie odbywa się za pomocą następującej operacji:

$${(0000\text{XXXX} \ll N)}_{(2)} \vee {000\text{\textbf{SS}}\text{\textbf{TTT}}}_{(2)},$$
gdzie $\text{XXXX}_{(2)}$ jest tagowaną liczbą, 
$N=4 \text{ lub } 5$, 
$\text{\textbf{SS}}_{(2)}$ są dodatkowymi bitami znakującymi rozmiar argumentu, a
$\text{\textbf{TTT}}_{(2)}$ jest danym tagiem.

\begin{longtable}{|c|c|p{9cm}|}
\hline

\multicolumn{2}{|c|}{\textbf{Tag}} & \multirow{2}{*}{\textbf{Typ}} \\
\cline{1-2}
\textbf{binarnie} & \textbf{dziesiętnie} & \\
\hline
\endfirsthead

000 & 0 & uniwersalny indeks, np. do tablicy stałych \\
\hline
001 & 1 & liczba całkowita \\ 
\hline
010 & 2 & indeks do tablicy atomów \\
\hline
011 & 3 & numer rejestru X maszyny wirtualnej \\
\hline
100 & 4 & numer rejestru Y maszyny wirtualnej \\
\hline
101 & 5 & etykieta, używana w funkcjach skoku \\
\hline
111 & 7 & złożone wyrażenie (np. lista, liczba zmiennoprzecinkowa). Obecna wersja kompilatora generuje tylko wartość 4 dla tego tagu, oznaczającą że złożone wyrażenie znajduje się w tablicy stałych. Indeks wyrażenia w tablicy stałych jest kolejnym argumentem zapisanym w postaci uniwersalnego indeksu. \\
\hline

\caption{Tagi typów danych w pliku ze skompilowanym modułem} 
\label{table:codeTags} \\
\end{longtable}

Jeżeli tagowana liczba jest nieujemna, mniejsza od 16 (możliwe jest zapisanie jej przy użyciu 4 bitów) to argument jest zapisany przy użyciu jednego bajtu a jego postać binarna to:
$$ \text{X}_1\text{X}_2\text{X}_3\text{X}_4\mathbf{0}\text{\textbf{TTT}}_{(2)}, $$
gdzie ${\text{X}_1\text{X}_2\text{X}_3\text{X}_4}_{(2)}$ to tagowana liczba, $\text{X}_1$ jest jej najbardziej znaczącym bitem, a $\text{TTT}_{(2)}$ to tag danego typu argumentu.

Na przykład, atom, który w tablicy atomów modułu ma indeks $2_{10} = 10_{2}$, po zakodowaniu będzie miał postać:
$$0010\mathbf{0010}_{2} = 22_{16} = 34_{10}.$$

W przypadku, gdy liczba jest nieujemna, mniejsza lub równa 16, a mniejsza od 2048 (możliwe jest jej zapisanie przy użyciu 11 bitów), argument jest zapisany przy użyciu dwóch bajtów, których postać binarna to:
$$  {\text{X}_1\text{X}_2\text{X}_3\mathbf{01}\text{\textbf{TTT}} \enskip \text{X}_4\text{X}_5\text{X}_6\text{X}_7\text{X}_8\text{X}_9\text{X}_{10}\text{X}_{11}}_{(2)}, $$
gdzie ${\text{X}_1 ... \text{X}_{11}}_{(2)}$ to tagowana liczba, $\text{X}_1$ jest jej najbardziej znaczącym bitem, a ${\text{TTT}}_{(2)}$ to tag danego typu argumentu.

Na przykład, liczba całkowita $565_{10} = {010 \enskip 00110101}_{2}$ po zakodowaniu będzie miała postać:
$${010\mathbf{01001} \enskip 00110101}_{2} = 4935_{16} = 18741_{10}.$$

Jeżeli argument jest liczbą ujemną lub dodatnią wymagającą w zapisie dwójkowym więcej niż 11 bitów to liczba taka zapisywana jest binarnie w kodzie uzupełnień do dwóch (U2) poprzedzona odpowiednim nagłówkiem.

Jeżeli zakodowaną liczbę można zapisać na nie więcej niż 8 bajtach, to nagłówek ma następującą postać:

$$ {\text{N}_1\text{N}_2\text{N}_3 \mathbf{11} \text{\textbf{TTT}}}_{(2)}, $$
gdzie ${\text{N}_1\text{N}_2\text{N}_3}_{(2)}$ to rozmiar argumentu w bajtach pomniejszony o 2 (jeżeli argument jest liczbą ujemną zajmującą 1 bajt to powinien on zostać dopełniony do 2 bajtów), $\text{N}_1$ jest jego najbardziej znaczącym bitem, a $\text{TTT}_{(2)}$ to tag danego typu argumentu.

Na przykład, aby zapisać na dwóch bajtach liczbę $-21_{10} = {11111111 \enskip 11101011}_{U2}$, jej postać binarną należy poprzedzić nagłówkiem:

$${000\mathbf{11001}}_{2} = 19_{16} = 25_{10}.$$

Jeżeli do zapisania liczby w kodzie uzupełnień do dwóch potrzeba przynajmniej 9 bajtów, wtedy nagłówek jest dwubajtowy i ma postać:

$$ {11111\text{\textbf{TTT}} \enskip \text{N}_1\text{N}_2\text{N}_3\text{N}_4 \mathbf{0000}}_{(2)}, $$
gdzie ${\text{N}_1\text{N}_2\text{N}_3\text{N}_4}_{(2)}$ to rozmiar argumentu w bajtach pomniejszony o 9, $\text{N}_1$ jest jego najbardziej znaczącym bitem, a $\text{TTT}_{(2)}$ to tag danego typu argumentu.

Na przykład, w celu zapisania liczby $2^{(15 \times 8)-1}-1$ na 15 bajtach, należy zapis tej liczby w kodzie U2 poprzedzić następującym nagłówkiem:

$${11111\mathbf{001} \enskip 0110\mathbf{0000}}_{2} = \text{F}960_{16} = 63840_{10}.$$

\section{Lista instrukcji}
\label{sec:opsOps}
%---------------------------------------------------------------------------
W tabeli \ref{tab:ops} zawarto listę instrukcji rozumianych przez maszynę wirtualną BEAM wraz z jednobajtowym kodem operacji, listą jej argumentów i krótkim opisem działania.

Na liście oznaczono operacje, które zostały zaimplementowane w maszynie wirtualnej opisywanej w pracy. Brak implementacji poszczególnych instrukcji podyktowanym jest brakiem wsparcia pewnych funkcjonalności lub typów danych w maszynie.

Instrukcje nieużywane przez kompilator Erlanga w wersji R16 zostały pominięte.

\begin{longtable}{|c|c|p{5cm}|p{6.75cm}|c|}
\hline

\multicolumn{2}{|c|}{\textbf{Kod operacji}} & \multirow{2}{*}{\textbf{Nazwa operacji i jej argumenty}} & \multirow{2}{*}{\textbf{Opis operacji i uwagi}} & \multirow{2}{*}{\textbf{Jest?}} \\
\cline{1-2}
\textbf{hex} & \textbf{dec} & & & \\
\hline
\endhead

01 & 1 & \texttt{label Lbl} & Wprowadza lokalną dla danego modułu etykietę identyfikującą aktualne miejsce w kodzie. & \cmark\\
\hline
02 & 2 & \texttt{func\_info M F A} & Definiuje funkcję \texttt{F}, w module \texttt{M} o arności \texttt{A} na jej początku. Instrukcja używana jest do generacji wyjątku \texttt{function\_clause} dla funkcji, którą definiuje.& \cmark\\
\hline
03 & 3 & \texttt{int\_code\_end} & Oznacza koniec kodu. & \cmark \\
\hline
04 & 4 & \texttt{call Arity Lbl} & Wywołuje funkcję o arności \texttt{Arity} znajdującą się pod etykietą \texttt{Lbl}. Zapisuje następną instrukcję jako adres powrotu (wskaźnik \textbf{CP}). & \cmark \\
\hline
05 & 5 & \texttt{call\_last Arity Lbl Dest} & Wywołuje rekurencyjną ogonowo funkcję o arności \texttt{Arity} znajdującą się pod etykietą \texttt{Lbl}. Nie zapisuje adresu powrotu. Przed wywołaniem zwalnia \texttt{Dest}+1 słów pamięci na stosie. & \cmark \\
\hline
06 & 6 & \texttt{call\_only Arity Lbl} & Wywołuje rekurencyjną ogonowo funkcję o arności \texttt{Arity} znajdującą się pod etykietą \texttt{Lbl}. Nie zapisuje adresu powrotu. & \cmark \\
\hline
07 & 7 & \texttt{call\_ext Arity Dest} & Wywołuje zewnętrzną funkcję o arności \texttt{Arity} mającą indeks \texttt{Dest} w tablicy funkcji zewnętrznych. Zapisuje następną instrukcję jako adres powrotu (wskaźnik \textbf{CP}). & \cmark \\
\hline
08 & 8 & \texttt{call\_ext\_last Arity Des Dea} & Wywołuje rekurencyjną ogonowo zewnętrzną funkcję o arności \texttt{Arity} mającą indeks \texttt{Des} w tablicy funkcji zewnętrznych. Nie zapisuje adresu powrotu. Przed wywołaniem zwalnia \texttt{Dea}+1 słów pamięci na stosie. Przywraca wskaźnik \textbf{CP} ze stosu. & \cmark \\
\hline
09 & 9 & \texttt{bif0 Bif Reg} & Wywołuje wbudowaną funkcję \texttt{Bif/0}. Wynik zapisywany jest w rejestrze \texttt{Reg}. & \xmark \\
\hline
0A & 10 & \texttt{bif1 Bif Arg Reg} & Wywołuje wbudowaną funkcję \texttt{Bif/1} z argumentem \texttt{Arg}. Wynik zapisywany jest w rejestrze \texttt{Reg}. & \xmark \\
\hline
0B & 11 & \texttt{bif2 Bif Arg1 Arg2 Reg} & Wywołuje wbudowaną funkcję \texttt{Bif/2} z argumentami \texttt{Arg1}, \texttt{Arg2}. Wynik zapisywany jest w rejestrze \texttt{Reg}. & \xmark \\
\hline
0C & 12 & \texttt{allocate StackN Live} & Alokuje miejsce dla \texttt{StackN} słów na stosie. Używanych jest \texttt{Live} rejestrów \textbf{X}, gdyby w trakcie alokacji konieczne było uruchomienie \emph{garbage collectora}. Zapisuje \textbf{CP} na stosie. & \cmark \\
\hline
0D & 13 & \texttt{allocate\_heap StackN HeapN Live} & Alokuje miejsce dla \texttt{StackN} słów na stosie. Upewnia się że na stercie jest \texttt{HeapN} wolnych słów. Używanych jest \texttt{Live} rejestrów \textbf{X}, gdyby w trakcie alokacji konieczne było uruchomienie \emph{garbage collectora}. Zapisuje \textbf{CP} na stosie. & \cmark \\
\hline
0E & 14 & \texttt{allocate\_zero StackN Live} & Tak jak \texttt{allocate/2}, ale zaalokowana pamięć jest wyzerowana. & \cmark \\
\hline
0F & 15 & \texttt{allocate\_heap\_zero SN HN L} & Tak jak \texttt{allocate\_heap/3}, ale zaalokowana pamięć jest wyzerowana. & \xmark \\
\hline
10 & 16 & \texttt{test\_heap HN L} & Upewnia się że na stercie jest \texttt{HN} wolnych słów. Używanych jest \texttt{L} rejestrów \textbf{X}, gdyby w trakcie konieczne było uruchomienie \emph{garbage collectora}. & \cmark\\
\hline
11 & 17 & \texttt{init N} & Zeruje \texttt{N}-te słowo na stosie. Instrukcja poprzednio nazywała się \texttt{kill} i jako taka może jeszcze występować w pewnych miejscach. & \cmark \\
\hline
12 & 18 & \texttt{deallocate N} & Przywraca \textbf{CP} ze stosu i dealokuje \texttt{N}+1 słów ze stosu. & \cmark \\
\hline
13 & 19 & \texttt{return} & Wraca do adresu zapisanego we wskaźniku \textbf{CP}. & \cmark \\
\hline
14 & 20 & \texttt{send} & Wysyła wiadomość z rejestru \textbf{X1} do procesu w rejestrze \textbf{X0}. & \xmark \\
\hline
15 & 21 & \texttt{remove\_message} & Usuwa aktualną wiadomość z kolejki wiadomości. Zapisuje wskaźnik do niej w rejestrze \textbf{X0}. Usuwa aktywne przeterminowanie (\emph{timeout}). & \xmark \\
\hline
16 & 22 & \texttt{timeout} & Resetuje wskaźnik \textbf{SAVE}. Czyści flagę przeterminowania. & \cmark \\
\hline
17 & 23 & \texttt{loop\_rec Lbl Src} & Zapisuje kolejną wiadomość w kolejce wiadomości \texttt{Src} w rejestrze \textbf{R0}. Jeśli jest pusta wykonuje skok do etykiety \texttt{Lbl}. & \xmark  \\
\hline
18 & 24 & \texttt{loop\_rec\_end Lbl} & Ustawia wskaźnik \textbf{SAVE} na kolejną wiadomość w kolejce wiadomości i wykonuje skok do etykiety \texttt{Lbl}. & \xmark \\
\hline
19 & 25 & \texttt{wait Lbl} & Zawiesza proces aż do otrzymania wiadomości, który zostanie wznowiony na początku bloku \texttt{receive} w etykiecie \texttt{Lbl}. & \xmark \\
\hline
1A & 26 & \texttt{wait\_timeout Lbl T} & Zawiesza proces jak \texttt{wait}. Ustawia przeterminowanie \texttt{T} i zapisuje następną instrukcję, która zostanie wykonana jeśli przeterminowanie się zrealizuje. & \cmark \\
\hline
27 & 39 & \texttt{is\_lt Lbl Arg1 Arg2} & Porównuje \texttt{Arg1} z \texttt{Arg2} i wykonuje skok do \texttt{Lbl} jeśli \texttt{Arg1} jest większe lub równe od \texttt{Arg2}. & \xmark \\
\hline
28 & 40 & \texttt{is\_ge Lbl Arg1 Arg2} & Porównuje \texttt{Arg1} z \texttt{Arg2} i wykonuje skok do \texttt{Lbl} jeśli \texttt{Arg1} jest mniejsze \texttt{Arg2}. & \xmark \\
\hline
29 & 41 & \texttt{is\_eq Lbl Arg1 Arg2} & Porównuje \texttt{Arg1} z \texttt{Arg2} i wykonuje skok do \texttt{Lbl} jeśli \texttt{Arg1} jest arytmetycznie różne od \texttt{Arg2}. & \xmark \\
\hline
2A & 42 & \texttt{is\_ne Lbl Arg1 Arg2} & Porównuje \texttt{Arg1} z \texttt{Arg2} i wykonuje skok do \texttt{Lbl} jeśli \texttt{Arg1} jest arytmetycznie równe \texttt{Arg2}. & \xmark \\
\hline
2B & 43 & \texttt{is\_eq\_exact Lbl Arg1 Arg2} & Porównuje \texttt{Arg1} z \texttt{Arg2} i wykonuje skok do \texttt{Lbl} jeśli \texttt{Arg1} jest różne \texttt{Arg2}. & \cmark \\
\hline
2C & 44 & \texttt{is\_ne\_exact Lbl Arg1 Arg2} & Porównuje \texttt{Arg1} z \texttt{Arg2} i wykonuje skok do \texttt{Lbl} jeśli \texttt{Arg1} jest równe \texttt{Arg2}. & \xmark \\
\hline
2D & 45 & \texttt{is\_integer Lbl Arg1} & Sprawdza typ \texttt{Arg1} i skacze do \texttt{Lbl} jeśli nie jest on liczbą całkowitą. & \xmark \\
\hline
2E & 46 & \texttt{is\_float Lbl Arg1} & Sprawdza typ \texttt{Arg1} i skacze do \texttt{Lbl} jeśli nie jest on liczbą rzeczywistą. & \xmark \\
\hline
2F & 47 & \texttt{is\_number Lbl Arg1} & Sprawdza typ \texttt{Arg1} i skacze do \texttt{Lbl} jeśli nie jest on liczbą. & \xmark \\
\hline
30 & 48 & \texttt{is\_atom Lbl Arg1} & Sprawdza typ \texttt{Arg1} i skacze do \texttt{Lbl} jeśli nie jest on atomem. & \xmark \\
\hline
31 & 49 & \texttt{is\_pid Lbl Arg1} & Sprawdza typ \texttt{Arg1} i skacze do \texttt{Lbl} jeśli nie jest on identyfikatorem procesu. & \xmark \\
\hline
32 & 50 & \texttt{is\_reference Lbl Arg1} & Sprawdza typ \texttt{Arg1} i skacze do \texttt{Lbl} jeśli nie jest on referencją. & \xmark \\
\hline
33 & 51 & \texttt{is\_port Lbl Arg1} & Sprawdza typ \texttt{Arg1} i skacze do \texttt{Lbl} jeśli nie jest on portem. & \xmark \\
\hline
34 & 52 & \texttt{is\_nil Lbl Arg1} & Sprawdza typ \texttt{Arg1} i skacze do \texttt{Lbl} jeśli nie jest on znacznikiem końca listy (\textbf{NIL}). & \cmark \\
\hline
35 & 53 & \texttt{is\_binary Lbl Arg1} & Sprawdza typ \texttt{Arg1} i skacze do \texttt{Lbl} jeśli nie jest on zmienną binarną. & \xmark \\
\hline
37 & 55 & \texttt{is\_list Lbl Arg1} & Sprawdza typ \texttt{Arg1} i skacze do \texttt{Lbl} jeśli nie jest on ani listą ani znacznikiem końca listy. & \xmark \\
\hline
38 & 56 & \texttt{is\_nonempty\_list Lbl Arg1} & Sprawdza typ \texttt{Arg1} i skacze do \texttt{Lbl} jeśli nie jest on niepustą listą. & \cmark \\
\hline
39 & 57 & \texttt{is\_tuple Lbl Arg1} & Sprawdza typ \texttt{Arg1} i skacze do \texttt{Lbl} jeśli nie jest on krotką. & \cmark \\
\hline
3A & 58 & \texttt{test\_arity Lbl Arg1 Arity} & Sprawdza arność krotki \texttt{Arg1} i skacze do \texttt{Lbl} jeśli nie jest ona równa \texttt{Arity}. & \cmark \\
\hline
3B & 59 & \texttt{select\_val Arg Lbl Dest} & Skacze do etykiety \texttt{Dest[Arg]}. Jeśli nie istnieje skacze do \texttt{Lbl}. & \xmark \\
\hline
3C & 60 & \texttt{select\_tuple\_arity Tuple Lbl Dest} & Sprawdza arność krotki \texttt{Tuple} i skacze do etykiety \texttt{Dest[Arity]}. Jeśli etykieta nie istnieje skacze do \texttt{Lbl}. & \xmark \\
\hline
3D & 61 & \texttt{jump Lbl} & Skacze do etykiety \texttt{Lbl}. & \cmark \\
\hline
3E & 62 & \texttt{catch Dest Lbl} & Tworzy nowy blok \texttt{catch}. Zapisuje etykietę \texttt{Lbl} na \texttt{Dest} miejscu na stosie. & \xmark \\
\hline
3F & 63 & \texttt{catch\_end Dest} & Kończy blok \texttt{catch}. Wymazuje etykietę na miejscu \texttt{Dest} na stosie. & \xmark  \\
\hline
40 & 64 & \texttt{move Src Dest} & Przenosi wartość z \texttt{Src} do rejestru \texttt{Dest}. & \cmark \\
\hline
41 & 65 & \texttt{get\_list Src Hd Tail} & Umieszcza głowę listy \texttt{Src} w rejestrze \texttt{Hd} i jej ogon w rejestrze \texttt{Tail}. & \cmark \\
\hline
42 & 66 & \texttt{get\_tuple\_element Src Elem Dest} & Umieszcza element \texttt{Elem} krotki \texttt{Src} w rejestrze \texttt{Dest}. & \cmark \\
\hline
43 & 67 & \texttt{set\_tuple\_element Elem Tuple Pos} & Umieszcza element \texttt{Elem} w krotce \texttt{Tuple} na pozycji \texttt{Pos}. & \xmark \\
\hline
45 & 69 & \texttt{put\_list Hd Tail Dest} & Tworzy komórkę listy \texttt{[Hd|Tail]} na szczycie sterty i umieszcza ją w rejestrze \texttt{Dest}. & \cmark \\
\hline
46 & 70 & \texttt{put\_tuple Dest Arity} & Tworzy krotkę o arności \texttt{Arity} na szczycie sterty i umieszcza ją w rejestrze \texttt{Dest}. & \cmark \\
\hline
47 & 71 & \texttt{put Arg} & Umieszcza \texttt{Arg} na szczycie sterty. & \cmark \\
\hline
48 & 72 & \texttt{badmatch Arg} & Rzuca wyjątek \texttt{badmatch} z argumentem \texttt{Arg}. & \xmark \\
\hline
49 & 73 & \texttt{if\_end} & Rzuca wyjątek \texttt{if\_clause}. & \xmark \\
\hline
4A & 74 & \texttt{case\_end Arg} & Rzuca wyjątek \texttt{case\_clause} z argumentem \texttt{Arg}. & \xmark \\
\hline
4B & 75 & \texttt{call\_fun Arity} & Woła obiekt funkcyjny o arności \texttt{Arity}. Zakłada, że argumenty znajdują się w rejestrach \textbf{X0}...\textbf{X(\texttt{Arity-1})}, a lambda w rejestrze \textbf{X(\texttt{Arity})}. Zapisuje następną instrukcję we wskaźniku \textbf{CP}. & \xmark \\
\hline
4D & 77 & \texttt{is\_function Lbl Arg1} & Sprawdza typ argumentu \texttt{Arg1} i skacze do \texttt{Lbl} jeśli nie jest on funkcją. & \xmark \\
\hline
4E & 78 & \texttt{call\_ext\_only Arity Lbl} & Wywołuje rekurencyjną ogonowo zewnętrzną funkcję o arności \texttt{Arity} mającą indeks \texttt{Lbl} w tablicy funkcji zewnętrznych. Nie zapisuje adresu powrotu. & \cmark \\
\hline
59 & 89 & \texttt{bs\_put\_integer Fail Size Unit Flags Src} & Umieszcza liczbę całkowitą w utworzonym wcześniej kontekście zmiennej binarnej, zajmując w nim rozmiar \texttt{Size} w jednostkach \texttt{Unit}. Źródłowa liczba znajduje się w rejestrze \texttt{Src}. Flagi dotyczące sposobu umieszczenia zmiennej w kontekście znajdują się w liczbie całkowitej \texttt{Flags}. Adres skoku \texttt{Fail} jest nieużywany. & \xmark  \\
\hline
5A & 90 & \texttt{bs\_put\_binary  Fail Size Unit Flags Src} & Umieszcza zmienną binarną w utworzonym wcześniej kontekście zmiennej binarnej, zajmując w nim rozmiar \texttt{Size} w jednostkach \texttt{Unit}. Źródłowa zmienna znajduje się w rejestrze \texttt{Src}. Flagi dotyczące sposobu umieszczenia zmiennej w kontekście znajdują się w liczbie całkowitej \texttt{Flags}. Adres skoku \texttt{Fail} jest nieużywany. & \xmark  \\
\hline
5B & 91 & \texttt{bs\_put\_float Fail Size Unit Flags Src} & Umieszcza liczbę zmiennoprzecinkową w utworzonym wcześniej kontekście zmiennej binarnej, zajmując w nim rozmiar \texttt{Size} w jednostkach \texttt{Unit}. Źródłowa liczba znajduje się w rejestrze \texttt{Src}. Flagi dotyczące sposobu umieszczenia liczby zmiennoprzecinkowej w kontekście znajdują się w liczbie całkowitej \texttt{Flags}. Adres skoku \texttt{Fail} jest nieużywany. & \xmark  \\
\hline
5C & 92 & \texttt{bs\_put\_string Size Bytes} & Bezpośrednio umieszcza bajty w utworzonym wcześniej kontekście zmiennej binarnej. Źródłowe bajty wskazywane są przez wskaźnik \texttt{Bytes} a ich liczba do przekopiowania to \texttt{Size}. & \xmark \\
\hline
5E & 94 & \texttt{fclearerror} &  Czyści flagę błędu zmiennoprzecinkowego, jeśli jest ustawiona. & \xmark \\
\hline
5F & 95 & \texttt{fcheckerror Arg0} &  Sprawdza czy \texttt{Arg0} zawiera wartość \texttt{NaN} lub nieskończoność. Jeśli tak, rzuca wyjątek \texttt{badarith}. & \xmark \\
\hline
60 & 96 & \texttt{fmove Arg0 Arg1} & Kopiuje wartość \texttt{Arg0} do \texttt{Arg1}. & \xmark  \\
\hline
61 & 97 & \texttt{fconv Arg0 Arg1} & Konwertuje wartość spod \texttt{Arg0} na liczbę zmiennoprzecinkową i umieszcza ją w rejestrze \texttt{Arg1}. & \xmark \\
\hline
62 & 98 & \texttt{fadd Arg0 Arg1 Arg2 Arg3} & Zapisuje w rejestrze \texttt{Arg3} wynik dodawania \texttt{Arg1} do \texttt{Arg2}. Argument \texttt{Arg0} jest nieużywany. & \xmark  \\
\hline
63 & 99 & \texttt{fsub Arg0 Arg1 Arg2 Arg3} & Zapisuje w rejestrze \texttt{Arg3} wynik odejmowania \texttt{Arg2} od \texttt{Arg1}. Argument \texttt{Arg0} jest nieużywany. & \xmark \\
\hline
64 & 100 & \texttt{fmul Arg0 Arg1 Arg2 Arg3} & Zapisuje w rejestrze \texttt{Arg3} wynik mnożenia \texttt{Arg1} przez \texttt{Arg2}. Argument \texttt{Arg0} jest nieużywany. & \xmark  \\
\hline
65 & 101 & \texttt{fdiv Arg0 Arg1 Arg2 Arg3} & Zapisuje w rejestrze \texttt{Arg3} wynik dzielenia \texttt{Arg1} przez \texttt{Arg2}. Argument \texttt{Arg0} jest nieużywany. & \xmark  \\
\hline
66 & 102 & \texttt{fnegate Arg0 Arg1 Arg2} & Zapisuje ujemną wartość z rejestru \texttt{Arg1} w rejestrze \texttt{Arg2}. Argument \texttt{Arg0} jest nieużywany. & \xmark \\
\hline
67 & 103 & \texttt{make\_fun2 N} & Odczytuje wpis o indeksie \texttt{N} w tablicy lambd modułu i umieszcza go w rejestrze \textbf{X0}. & \xmark  \\
\hline
68 & 104 & \texttt{try Dest Label} & Tak jak instrukcja \texttt{catch/2}.  & \xmark  \\
\hline
69 & 105 & \texttt{try\_end Dest} & Kończy blok \texttt{catch}. Wymazuje etykietę na miejscu \texttt{Dest} na stosie. Jeśli nie ma zapisanej wartości w rejestrze \textbf{R0} oznacza to że w bloku \texttt{catch} złapano wyjątek. W takim przypadku dokonywane jest przepisanie wartości rejestrów: \textbf{R0} = \textbf{X1}, \textbf{X1} = \textbf{X2} i \textbf{X2} = \textbf{X3}. & \xmark \\
\hline
6A & 106 & \texttt{try\_case Dest} & Jak instrukcja \texttt{try\_end/1}. & \xmark  \\
\hline
6B & 107 & \texttt{try\_case\_end Reason} &  Rzuca wyjątek \texttt{try\_clause} z argumentem \texttt{Reason}. & \xmark \\
\hline
6C & 108 & \texttt{raise Stacktrace Reason} & Rzuca wyjątek \texttt{Reason} ze stosem wywołań \texttt{Stacktrace}. & \xmark \\
\hline
6D & 109 & \texttt{bs\_init2 Fail Sz Words Regs Flags Dst} & Inicjalizuje miejsce dla zmiennej binarnej o rozmiarze \texttt{Sz}. Zapewnia, że oprócz tego rozmiaru po inicjalizacji dostępne będzie dodatkowo \texttt{Words} słów maszynowych. \texttt{Regs} oznacza liczbę aktywnych rejestrów, w razie gdyby w trakcie inicjalizacji konieczne było uruchomienie \emph{garbage collectora}. Adres skoku \texttt{Fail} oraz opcje \texttt{Flags} są aktualnie nieużywane. & \xmark  \\
\hline
6F & 111 & \texttt{bs\_add Fail S1 S2 Unit Dst} & Oblicza sumę bitów w \texttt{S1} i liczby jednostek \texttt{Unit} w \texttt{S2}. Wynik przechowuje w \texttt{Dst}. W przypadku niepowodzenia wykonuje skok do etykiety \texttt{Fail}. & \xmark  \\
\hline
70 & 112 & \texttt{apply N} & Znajduje adres początku funkcji zapisanej w rejestrze\textbf{X(\texttt{N+1})}, w module zapisanym w rejestrze \textbf{X(\texttt{N})} o arności \texttt{N} i skacze do tego adresu. Zapisuje następną instrukcję we wskaźniku \textbf{CP}. & \xmark \\
\hline 
71 & 113 & \texttt{apply\_last N Dea} & Skacze do zewnętrznej funkcji tak jak instrukcja \texttt{apply/1}. Ściąga wartość wskaźnika \textbf{CP} ze stosu. Zwalnia \texttt{Dea} miejsc na szczycie stosu. & \xmark \\
\hline
72 & 114 & \texttt{is\_boolean/2} & Sprawdza typ \texttt{Arg1} i skacze do \texttt{Lbl} jeśli nie jest on ani atomem \texttt{true} ani \texttt{false}. & \xmark \\
\hline
73 & 115 & \texttt{is\_function2 Lbl Arg1 Arity} & Sprawdza typ \texttt{Arg1} i skacze do \texttt{Lbl} jeśli nie jest on funkcją o arności \texttt{Arity}. & \xmark \\
\hline
74 & 116 & \texttt{bs\_start\_match2 Fail Bin X Y Dst} & Sprawdza czy \texttt{Bin} jest kontekstem zmiennej binarnej z odpowiednią liczbą miejsc do zapisania tymczasowych przesunięć bitowych (\emph{offsetów}) określonych przez liczbę \texttt{Y}. Jeśli aktualna liczba miejsc jest za mała tworzony jest nowy kontekst porównywania zmiennej binarnej z odpowiednią liczbą miejsc, który zostanie zapisany w miejscu \texttt{Dst}. W przypadku niepowodzenia dokonany zostanie skok do etykiety \texttt{Fail}. \texttt{X} oznacza liczbę aktywnych rejestrów, w razie gdyby w trakcie operacji konieczne było uruchomienie \emph{garbage collectora}. & \xmark \\
\hline
75 & 117 & \texttt{bs\_get\_integer2 Fail Ms Live Sz Unit Flags Dst} & Pobiera liczbę całkowitą z kontekstu zmiennej binarnej \texttt{Ms} o rozmiarze \texttt{Sz} w jednostkach \texttt{Unit}. Wynik zapisywany jest do \texttt{Dst}. Opcje operacji zapisywane są w liczbie całkowitej \texttt{Flags}.  W przypadku niepowodzenia wykonywany jest skok do etykiety \texttt{Fail}. \texttt{Live} oznacza liczbę aktywnych rejestrów, w razie gdyby w trakcie operacji konieczne było uruchomienie \emph{garbage collectora}. & \xmark  \\
\hline
76 & 118 & \texttt{bs\_get\_float2 Fail Ms Live Sz Unit Flags Dst} & Pobiera liczbę zmiennoprzecinkową z kontekstu zmiennej binarnej \texttt{Ms} o rozmiarze \texttt{Sz} w jednostkach \texttt{Unit}. Wynik zapisywany jest do \texttt{Dst}. Opcje operacji zapisywane są w liczbie całkowitej \texttt{Flags}.  W przypadku niepowodzenia wykonywany jest skok do etykiety \texttt{Fail}. \texttt{Live} oznacza liczbę aktywnych rejestrów, w razie gdyby w trakcie operacji konieczne było uruchomienie \emph{garbage collectora}. & \xmark  \\
\hline
77 & 119 & \texttt{bs\_get\_binary2 Fail Ms Live Sz Unit Flags Dst} & Pobiera zmienną binarną z kontekstu zmiennej binarnej \texttt{Ms} o rozmiarze \texttt{Sz} w jednostkach \texttt{Unit}. Wynik zapisywany jest do \texttt{Dst}. Opcje operacji zapisywane są w liczbie całkowitej \texttt{Flags}.  W przypadku niepowodzenia wykonywany jest skok do etykiety \texttt{Fail}. \texttt{Live} oznacza liczbę aktywnych rejestrów, w razie gdyby w trakcie operacji konieczne było uruchomienie \emph{garbage collectora}.  & \xmark \\
\hline
78 & 120 & \texttt{bs\_skip\_bits2 Fail Ms Sz Unit Flags} & Pomija \texttt{Sz} jednostek wyrażonych w \texttt{Unit} z kontekstu zmiennej binarnej \texttt{Ms}. Opcje operacji wyrażone są za pomocą liczby całkowitej \texttt{Flags}. W przypadku niepowodzenia wykonywany jest skok do etykiety \texttt{Fail}.  & \xmark \\
\hline
79 & 121 & \texttt{bs\_test\_tail2 Fail Ms Bits} & Sprawdza, czy kontekst zmiennej binarnej \texttt{Ms} ma jeszcze dokładnie \texttt{Bits} niedopasowanych bitów. Wykonuje skok do etykiety \texttt{Fail} jeżeli tak nie jest. & \xmark   \\
\hline
7A & 122 & \texttt{bs\_save2 Reg Index} & Zapisuje aktualne przesunięcie bitowe \emph{offset} z kontekstu zmiennej binarnej zawartego w rejestrze \texttt{Reg} i zapisuje do jego tablicy \emph{offsetów}. & \xmark \\
\hline
7B & 123 & \texttt{bs\_restore2 Reg Index} & Odczytuje przesunięcie bitowe kontekstu zmiennej binarnej zapisanego w rejestrze \texttt{Reg} z indeksu \texttt{Index}. Zapisuje go jako aktualne dla tego kontekstu. & \xmark  \\
\hline
7C & 124 & \texttt{gc\_bif1 Lbl Live Bif Arg1 Reg} & Wywołuje funkcję wbudowaną \texttt{Bif/1} z argumentem \texttt{Arg1}. Wynik zapisuje w rejestrze \texttt{Reg}. W przypadku niepowodzenia skacze do etykiety \texttt{Lbl}. Uruchamia \emph{garbage collector} jeśli jest to konieczne, zachowując \texttt{Live} rejestrów \textbf{X}. & \xmark \\
\hline
7D & 125 & \texttt{gc\_bif2 Lbl Live Bif Arg1 Arg2 Reg} & Wywołuje funkcję wbudowaną \texttt{Bif/2} z argumentami \texttt{Arg1}, \texttt{Arg2}. Wynik zapisuje w rejestrze \texttt{Reg}. W przypadku niepowodzenia skacze do etykiety \texttt{Lbl}. Uruchamia \emph{garbage collector} jeśli jest to konieczne, zachowując \texttt{Live} rejestrów \textbf{X}.  & \cmark  \\
\hline
81 & 129 & \texttt{is\_bitstr Lbl Arg1} & Sprawdza typ \texttt{Arg1} i skacze do \texttt{Lbl} jeśli nie jest on ciągiem bitów. & \xmark \\
\hline
82 & 130 & \texttt{bs\_context\_to\_binary Reg} & Zamienia kontekst zmiennej binarnej znajdującej się w rejestrze \texttt{Reg} na właściwą zmienną binarną. & \xmark \\
\hline
83 & 131 & \texttt{bs\_test\_unit Fail Ms Unit} & Sprawdza czy rozmiar niedopasowanego jeszcze fragmentu kontekstu zmiennej binarnej \texttt{Ms} jest podzielny przez \texttt{Unit}. Jeżeli nie, wykonywany jest skok do etykiety \texttt{Fail}. & \xmark  \\
\hline
84 & 132 & \texttt{bs\_match\_string Fail Ms Bits Val} & Dokonuje porównania \texttt{Bits} bitów, począwszy od miejsca wskazywanego przez wskaźnik \texttt{Val} z kontekstem zmiennej binarnej \texttt{Ms}. Jeżeli porównywane wartości nie są równe dokonywany jest skok do etykiety \texttt{Fail}. & \xmark  \\
\hline
85 & 133 & \texttt{bs\_init\_writable} & Alokuje miejsce o rozmiarze \textbf{R0} na stercie procesu. Tworzy w zaalokowanym miejscu strukturę zmiennej binarnej. Dodatkowo tworzy wskaźnik do utworzonej struktury, który umieszczony zostanie w rejestrze \textbf{R0}. & \xmark  \\
\hline
86 & 134 & \texttt{bs\_append Fail Size Extra Live Unit Bin Flags Dst} & Dopisuje \texttt{Size} jednostek \texttt{Unit} do zmiennej binarnej \texttt{Bin} i zapisuje wynik do \texttt{Dst}. Jeśli nie ma wystarczająco dużej ilości miejsca, tworzona jest nowa struktura na stercie z odpowiednią ilością miejsca, powiększona dodatkowo o \texttt{Extra} słów maszynowych. W przypadku niepowodzenia wykonywany jest skok do etykiety \texttt{Fail}. Opcje operacji zapisane są w liczbie całkowitej \texttt{Flags}. \texttt{Live} oznacza liczbę aktywnych rejestrów, w razie gdyby w trakcie alokacji konieczne było uruchomienie \emph{garbage collectora}.  & \xmark \\
\hline
87 & 135 & \texttt{bs\_private\_append Fail Size Unit Bin Flags Dst} & Operacja ma działanie podobne do operacji \texttt{bs\_append/8} jednak w przypadku zbyt małej ilości miejsca dokonuje realokacji aktualnej zmiennej. & \xmark \\
\hline
88 & 136 & \texttt{trim N Remaining} & Redukuje stos o \texttt{N} słów, zachowując \textbf{CP} na jego szczycie. & \cmark \\
\hline
89 & 137 & \texttt{bs\_init\_bits Fail Sz Words Regs Flags Dst} & Alokuje zmienną binarną na stercie o rozmiarze \texttt{Sz} bitów. Jeżeli rozmiar nie jest podzielny przez 8, tworzony jest wskaźnik na strukturę z zapisaną ilością zajmowanych bitów. Wynik operacji zapisany jest do \texttt{Dst}. Upewnia się że na stercie jest dodatkowo \texttt{Words} słów maszynowych możliwych do zaalokowania. Opcje operacji zapisane są w liczbie całkowitej \texttt{Flags}. W razie niepowodzenia wykonywany jest skok do etykiety \texttt{Fail}. \texttt{Regs} oznacza liczbę aktywnych rejestrów, w razie gdyby w trakcei alokacji konieczne było uruchomienie \emph{garbage collectora}. & \xmark \\
\hline
8A & 138 & \texttt{bs\_get\_utf8 Fail Ms Arg2 Arg3 Dst} & Pobiera znak zapisany w UTF-8 z kontekstu zmiennej binarnej \texttt{Ms} i zapisuje go do \texttt{Dst}. W przypadku niepowodzenia wykonuje skok do etykiety \texttt{Fail}. Argumenty \texttt{Arg2} oraz \texttt{Arg3} są aktualnie nieużywane.  & \xmark \\
\hline
8B & 139 & \texttt{bs\_skip\_utf8 Fail Ms Arg2 Arg3} & Pomija znak zakodowany w UTF-8 w kontekście zmiennej binarnej \texttt{Ms}. W przypadku niepowodzenia wykonuje skok do etykiety \texttt{Fail}. Argumenty \texttt{Arg2} oraz \texttt{Arg3} są aktualnie nieużywane. & \xmark  \\
\hline
8C & 140 & \texttt{bs\_get\_utf16 Fail Ms Arg2 Flags Dst} & Pobiera znak zapisany w UTF-16 z kontekstu zmiennej binarnej \texttt{Ms}, używając opcji zapisanych w liczbie całkowitej \texttt{Flags} i zapisuje go do \texttt{Dst}. W przypadku niepowodzenia wykonuje skok do etykiety \texttt{Fail}. Argument \texttt{Arg2} jest aktualnie nieużywany. & \xmark  \\
\hline
8D & 141 & \texttt{bs\_skip\_utf16 Fail Ms Arg2 Flags} & Pomija znak zakodowany w UTF-16 w kontekście zmiennej binarnej \texttt{Ms}, używając opcji zapisanych w liczbie całkowitej \texttt{Flags}. W przypadku niepowodzenia wykonuje skok do etykiety \texttt{Fail}. Argument \texttt{Arg2} jest aktualnie nieużywany. & \xmark \\
\hline
8E & 142 & \texttt{bs\_get\_utf32 Fail Ms Live Flags Dst} & Pobiera znak zapisany w UTF-32 z kontekstu zmiennej binarnej \texttt{Ms}, używając opcji zapisanych w liczbie całkowitej \texttt{Flags} i zapisuje go do \texttt{Dst}. \texttt{Live} oznacza liczbę aktywnych rejestrów, w razie gdyby w trakcie operacji konieczne było uruchomienie \emph{garbage collectora}. W przypadku niepowodzenia wykonuje skok do etykiety \texttt{Fail}. & \xmark \\
\hline
8F & 143 & \texttt{bs\_skip\_utf32 Fail Ms Live Flags} & Pomija znak zakodowany w UTF-32 w kontekście zmiennej binarnej \texttt{Ms}, używając opcji zapisanych w liczbie całkowitej \texttt{Flags}. \texttt{Live} oznacza liczbę aktywnych rejestrów, w razie gdyby w trakcie operacji konieczne było uruchomienie \emph{garbage collectora}. W przypadku niepowodzenia wykonuje skok do etykiety \texttt{Fail}. & \xmark  \\
\hline
90 & 144 & \texttt{bs\_utf8\_size Fail Literal Dst} & Oblicza liczbę bajtów koniecznych do zapisania \texttt{Literal} w UTF-8 i zapisuje wynik do \texttt{Dst}. Argument \texttt{Fail} jest aktualnie nieużywany. & \xmark  \\
\hline
91 & 145 & \texttt{bs\_put\_utf8 Fail Flags Src} & Umieszcza znak zakodowany w UTF-8 znajdujący się w \texttt{Src} w aktualnym kontekście zmiennej binarnej. Argumenty \texttt{Fail} i \texttt{Flags} są aktualnie nieużywane. & \xmark  \\
\hline
92 & 146 & \texttt{bs\_utf16\_size Fail Literal Dst} & Oblicza liczbę bajtów koniecznych do zapisania \texttt{Literal} w UTF-16 i zapisuje wynik do \texttt{Dst}. Argument \texttt{Fail} jest aktualnie nieużywany. & \xmark  \\
\hline
93 & 147 & \texttt{bs\_put\_utf16 Fail Flags Literal} & Umieszcza znak zakodowany w UTF-16 znajdujący się w \texttt{Src} w aktualnym kontekście zmiennej binarnej z użyciem opcji zapisanych w liczbie całkowitej \texttt{Flags}. Argument \texttt{Fail} jest aktualnie nieużywany. & \xmark  \\
\hline
94 & 148 & \texttt{bs\_put\_utf32 Fail Flags Literal} &  Umieszcza znak zakodowany w UTF-32 znajdujący się w \texttt{Src} w aktualnym kontekście zmiennej binarnej z użyciem opcji zapisanych w liczbie całkowitej \texttt{Flags}. W przypadku niepowodzenia wykonuje skok do etykiety \texttt{Fail}. & \xmark  \\
\hline
95 & 149 & \texttt{on\_load} & Oznacza kod wykonywany przy ładowaniu modułu. & \xmark \\
\hline
96 & 150 & \texttt{recv\_mark Lbl} & Zapamiętuje aktualną wiadomość z kolejki oraz etykietę \texttt{Label} do instrukcji \texttt{loop\_rec/2}. & \xmark \\
\hline
97 & 151 & \texttt{recv\_set Lbl} & Jeśli etykieta \texttt{Lbl} wskazuje na instrukcję \texttt{loop\_rec/2} to przepisuje wiadomość zachowaną przez instrukcję \texttt{recv\_mark} do wskaźnika \textbf{SAVE}. & \xmark  \\
\hline
98 & 152 & \texttt{gc\_bif3 Lbl Live Bif Arg1 Arg2 Arg3 Reg} & Wywołuje funkcję wbudowaną \texttt{Bif/3} z argumentami \texttt{Arg1}, \texttt{Arg2}, \texttt{Arg3}. Wynik zapisuje w rejestrze \texttt{Reg}. W przypadku niepowodzenia skacze do etykiety \texttt{Lbl}. Uruchamia \emph{garbage collector} jeśli jest to konieczne, zachowując \texttt{Live} rejestrów \textbf{X}. & \xmark  \\
\hline
99 & 153 & \texttt{line N} & Znakuje aktualne miejsce jako linia o indeksie \texttt{N} w tablicy linii. & \xmark \\
\hline


\caption{Lista operacji maszyny wirtualnej BEAM} 
\label{tab:ops} \\
\end{longtable}