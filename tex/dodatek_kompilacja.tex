\chapter{Kompilacja kodu źródłowego}
\label{cha:erlangKompilacja}

Podrozdział opisuje kolejne kroki, z jakich składa się proces otrzymywania skompilowanego kodu pośredniego maszyny wirtualnej BEAM z kodu źródłowego napisanego w języku Erlang.

%---------------------------------------------------------------------------
\section{Wprowadzenie}

Jak zostało wspomniane w \ref{sec:erlangMaszyny}, program napisany w języku Erlang wykonywany jest na dedykowanej do tego celu maszynie wirtualnej.

Narzędzia przeznaczone do operacji opisanych w niniejszym rozdziale zostały napisane w języku Erlang i dostępne są w aplikacji \textbf{compiler} dostarczanej wraz z maszyną wirtualną BEAM.
%---------------------------------------------------------------------------
\section{Kod źródłowy}

\begin{lstlisting}[style=erlang, caption=Plik fac.erl, label=facERL]
-module(fac).

-export([fac/1]).
-define(ERROR, "Invalid argument").

-include("fac.hrl").

fac(#factorial{n=0, acc=Acc}) ->
    Acc;
fac(#factorial{n=N, acc=Acc}) ->
    fac(#factorial{n=N-1, acc=N*Acc});
fac(N) when is_integer(N) ->
    fac(#factorial{n=N});
fac(N) when is_binary(N) ->
    fac(binary_to_integer(N));
fac(_) ->
    {error, ?ERROR}.
\end{lstlisting}

\begin{lstlisting}[style=erlang, caption=Plik fac.hrl, label=facHRL]
-record(factorial, {n, acc=1}).
\end{lstlisting}

%---------------------------------------------------------------------------
\section{Preprocessing}

\begin{lstlisting}[style=erlang, caption=Moduł fac po pierwszym przetworzeniu, label=facP]
-file("fac.erl", 1).

-module(fac).

-export([fac/1]).

-file("fac.hrl", 1).

-record(factorial,{n,acc = 1}).

-file("fac.erl", 7).

fac(#factorial{n = 0,acc = Acc}) ->
    Acc;
fac(#factorial{n = N,acc = Acc}) ->
    fac(#factorial{n = N - 1,acc = N * Acc});
fac(N) when is_integer(N) ->
    fac(#factorial{n = N});
fac(N) when is_binary(N) ->
    fac(binary_to_integer(N));
fac(_) ->
    {error,"Invalid argument"}.
\end{lstlisting}

\begin{lstlisting}[style=erlang, caption=Moduł fac po drugim przetworzeniu, label=facE]
-file("fac.erl", 1).

-file("fac.hrl", 1).

-file("fac.erl", 7).

fac({factorial,0,Acc}) ->
    Acc;
fac({factorial,N,Acc}) ->
    fac({factorial,N - 1,N * Acc});
fac(N) when is_integer(N) ->
    fac({factorial,N,1});
fac(N) when is_binary(N) ->
    fac(binary_to_integer(N));
fac(_) ->
    {error,"Invalid argument"}.

module_info() ->
    erlang:get_module_info(fac).

module_info(X) ->
    erlang:get_module_info(fac, X).
\end{lstlisting}
%---------------------------------------------------------------------------
\section{Transformacje drzewa syntaktycznego}
\label{sec:compilationSyntaxtree}

\begin{lstlisting}[style=erlang, caption=Drzewo syntaktyczne modułu fac, label=facAE]
[{attribute,1,file,{"fac.erl",1}},
 {attribute,1,module,fac},
 {attribute,5,export,[{fac,1}]},
 {attribute,1,file,{"fac.hrl",1}},
 {attribute,1,record,
     {factorial,
         [{record_field,1,{atom,1,n}},
          {record_field,1,{atom,1,acc},{integer,1,1}}]}},
 {attribute,9,file,{"fac.erl",9}},
 {function,10,fac,1,
     [{clause,10,
          [{record,10,factorial,
               [{record_field,10,{atom,10,n},{integer,10,0}},
                {record_field,10,{atom,10,acc},{var,10,'Acc'}}]}],
          [],
          [{var,11,'Acc'}]},
      {clause,12,
          [{record,12,factorial,
               [{record_field,12,{atom,12,n},{var,12,'N'}},
                {record_field,12,{atom,12,acc},{var,12,'Acc'}}]}],
          [],
          [{call,13,
               {atom,13,fac},
               [{record,13,factorial,
                    [{record_field,13,
                         {atom,13,n},
                         {op,13,'-',{var,13,'N'},{integer,13,1}}},
                     {record_field,13,
                         {atom,13,acc},
                         {op,13,'*',{var,13,'N'},{var,13,'Acc'}}}]}]}]},
      {clause,14,
          [{var,14,'N'}],
          [[{call,14,{atom,14,is_integer},[{var,14,'N'}]}]],
          [{call,15,
               {atom,15,fac},
               [{record,15,factorial,
                    [{record_field,15,{atom,15,n},{var,15,'N'}}]}]}]},
      {clause,16,
          [{var,16,'N'}],
          [[{call,16,{atom,16,is_binary},[{var,16,'N'}]}]],
          [{call,17,
               {atom,17,fac},
               [{call,17,{atom,17,binary_to_integer},[{var,17,'N'}]}]}]},
      {clause,18,
          [{var,18,'_'}],
          [],
          [{tuple,19,[{atom,19,error},{string,19,"Invalid argument"}]}]}]},
 {eof,20}]
\end{lstlisting}
%---------------------------------------------------------------------------
\section{Kod pośredni (\emph{bytecode})}
\begin{lstlisting}[style=erlang, caption=\emph{Bytecode} modułu fac, label=facS]
{module, fac}.  %% version = 0

{exports, [{fac,1},{module_info,0},{module_info,1}]}.

{attributes, []}.

{labels, 11}.


{function, fac, 1, 2}.
  {label,1}.
    {line,[{location,"fac.erl",8}]}.
    {func_info,{atom,fac},{atom,fac},1}.
  {label,2}.
    {test,is_tuple,{f,4},[{x,0}]}.
    {test,test_arity,{f,4},[{x,0},3]}.
    {get_tuple_element,{x,0},0,{x,1}}.
    {get_tuple_element,{x,0},1,{x,2}}.
    {get_tuple_element,{x,0},2,{x,3}}.
    {test,is_eq_exact,{f,4},[{x,1},{atom,factorial}]}.
    {test,is_eq_exact,{f,3},[{x,2},{integer,0}]}.
    {move,{x,3},{x,0}}.
    return.
  {label,3}.
    {line,[{location,"fac.erl",11}]}.
    {gc_bif,'-',{f,0},4,[{x,2},{integer,1}],{x,0}}.
    {line,[{location,"fac.erl",11}]}.
    {gc_bif,'*',{f,0},4,[{x,2},{x,3}],{x,1}}.
    {test_heap,4,4}.
    {put_tuple,3,{x,2}}.
    {put,{atom,factorial}}.
    {put,{x,0}}.
    {put,{x,1}}.
    {move,{x,2},{x,0}}.
    {call_only,1,{f,2}}.
  {label,4}.
    {test,is_integer,{f,5},[{x,0}]}.
    {test_heap,4,1}.
    {put_tuple,3,{x,1}}.
    {put,{atom,factorial}}.
    {put,{x,0}}.
    {put,{integer,1}}.
    {move,{x,1},{x,0}}.
    {call_only,1,{f,2}}.
  {label,5}.
    {test,is_binary,{f,6},[{x,0}]}.
    {allocate,0,1}.
    {line,[{location,"fac.erl",15}]}.
    {call_ext,1,{extfunc,erlang,binary_to_integer,1}}.
    {call_last,1,{f,2},0}.
  {label,6}.
    {move,{literal,{error,"Invalid argument"}},{x,0}}.
    return.


{function, module_info, 0, 8}.
  {label,7}.
    {line,[]}.
    {func_info,{atom,fac},{atom,module_info},0}.
  {label,8}.
    {move,{atom,fac},{x,0}}.
    {line,[]}.
    {call_ext_only,1,{extfunc,erlang,get_module_info,1}}.


{function, module_info, 1, 10}.
  {label,9}.
    {line,[]}.
    {func_info,{atom,fac},{atom,module_info},1}.
  {label,10}.
    {move,{x,0},{x,1}}.
    {move,{atom,fac},{x,0}}.
    {line,[]}.
    {call_ext_only,2,{extfunc,erlang,get_module_info,2}}.
\end{lstlisting}

%---------------------------------------------------------------------------
\section{Plik binarny BEAM}
%---------------------------------------------------------------------------

Efektem przetworzenia kodu pośredniego, wyrażonego w postaci krotek, jest plik binarny w formacie IFF \cite{morrison1985ea}, w formacie zrozumiałym przez maszynę wirtualną BEAM. Maszyna ta wykorzystuje tego rodzaju pliki do ładowania kodu modułów do pamięci. Ich źródłem może być zarówno system plików na fizycznej maszynie, na której uruchomiony został BEAM, jak i inna maszyna wirtualna znajdująca się w tym samym klastrze \emph{Distributed Erlang}, co docelowa.

W tabeli \ref{table:beamFile} zaprezentowana została struktura pliku binarnego ze skompilowanym modułem.

\begin{longtable}{|c|c|c|c|c|c|c|c|c|c|c|c|c|c|c|c|c|c|}
\hline
         & \textbf{Oktet} & \multicolumn{8}{|c|}{\textbf{0}} & \multicolumn{8}{|c|}{\textbf{1}} \\
\hline
\textbf{Oktet} & \textbf{Bit} & \textbf{0} & \textbf{1} & \textbf{2} & \textbf{3} & \textbf{4} & \textbf{5} & \textbf{6} & \textbf{7} & \textbf{8} & \textbf{9} & \textbf{10} & \textbf{11} & \textbf{12} & \textbf{13} & \textbf{14} & \textbf{15}\\
\hline
\textbf{0} & \textbf{0} & \multicolumn{16}{|c|}{"FOR1"} \\[3ex]
\hline
\textbf{4} & \textbf{32} & \multicolumn{16}{|c|}{Rozmiar pliku bez pierwszych 8 bajtów}\\[3ex]
\hline
\textbf{8} & \textbf{64} & \multicolumn{16}{|c|}{"BEAM"} \\[3ex]
\hline
\textbf{12} & \textbf{96} & \multicolumn{16}{|c|}{Identyfikator fragmentu (\emph{chunk}) 1}\\[3ex]
\hline
\textbf{16} & \textbf{128} & \multicolumn{16}{|c|}{Rozmiar fragmentu 1} \\[3ex]
\hline
\textbf{20} & \textbf{160} & \multicolumn{16}{|c|}{Dane fragmentu 1} \\[10ex]
\hline
\textbf{...} & \textbf{...} & \multicolumn{16}{|c|}{Identyfikator fragmentu (\emph{chunk}) 2}\\[3ex]
\hline
\textbf{...} & \textbf{...} & \multicolumn{16}{|c|}{...} \\[10ex]
\hline
\caption{Struktura pliku modułu BEAM}
\label{table:beamFile} \\
\end{longtable}

Każdy plik binarny BEAM powinien zawierać przynajmniej 4 następujące fragmenty (\emph{chunki}).
Obok opisu każdego fragmentu, w nawiasie podano ciąg znaków będący jego identyfikatorem w binarnym pliku modułu:
\begin{itemize}
\item tablica atomów wykorzystywanych przez moduł (\texttt{Atom});
\item kod pośredni danego modułu (\texttt{Code});
\item tablica zewnętrznych funkcji używanych przez moduł (\texttt{ImpT});
\item tablica funkcji eksportowanych przez moduł (\texttt{ExpT}).
\end{itemize}

Ponadto, w pliku mogą znajdować się następujące fragmenty:
\begin{itemize}
\item tablica funkcji lokalnych dla danego modułu (\texttt{LocT});
\item tablica lambd wykorzystwanych przed moduł (\texttt{FunT});
\item tablica stałych wykorzystywanych przed moduł (\texttt{LitT});
\item lista atrybutów modułu (\texttt{Attr});
\item lista dodatkowych informacji o kompilacji modułu (\texttt{CInf)};
\item tablica linii kodu źródłowego modułu (\texttt{Line});
\item drzewo syntaktyczne modułu (\texttt{Abst}).
\end{itemize}

W przypadku każdego rodzaju fragmentu, obszar jaki zajmuje on w pliku jest zawsze wielokrotnością 4 bajtów. Nawet jeżeli nagłówek fragmentu, zawierający jego rozmiar, nie jest podzielny przez 4, obszar zaraz za danym fragmentem dopełniany jest zerami do pełnych 4 bajtów.

Warto zaznaczyć również, że sposób implementacji maszyny wirtualnej BEAM nie definiuje kolejności w jakiej poszczególne fragmenty powinny występować w pliku binarnym.

%---------------------------------------------------------------------------
\subsection{Tablica atomów}
%---------------------------------------------------------------------------
Tablica atomów zawiera listę wszystkich atomów, które używane są przez dany moduł. W trakcie ładowania kodu modułu przez maszynę wirtualną, atomy, które nie wystepowały we wcześniej załadowanych modułach, zostają wstawione do globalnej tablicy atomów (w postaci tablicy z hashowaniem).

Fragment piku binarnego z tablicą atomów reprezentowany jest przez napis \texttt{Atom}. Struktura danych fragmentu zaprezentowana jest w tabeli \ref{table:atomTable}.

\begin{longtable}{|c|c|c|c|c|c|c|c|c|c|c|c|c|c|c|c|c|c|}
\hline
         & \textbf{Oktet} & \multicolumn{8}{|c|}{\textbf{0}} & \multicolumn{8}{|c|}{\textbf{1}} \\
\hline
\textbf{Oktet} & \textbf{Bit} & \textbf{0} & \textbf{1} & \textbf{2} & \textbf{3} & \textbf{4} & \textbf{5} & \textbf{6} & \textbf{7} & \textbf{8} & \textbf{9} & \textbf{10} & \textbf{11} & \textbf{12} & \textbf{13} & \textbf{14} & \textbf{15}\\
\hline
\textbf{0} & \textbf{0} & \multicolumn{16}{|c|}{Ilość atomów w tablicy atomów} \\[3ex]
\hline
\textbf{4} & \textbf{32} & \multicolumn{8}{|c|}{Dł. atomu 1} & \multicolumn{8}{|c|}{Nazwa atomu 1 w ASCII}\\[3ex]
\hline
\textbf{...} & \textbf{...} & \multicolumn{8}{|c|}{Dł. atomu 2} & \multicolumn{8}{|c|}{Nazwa atomu 2 w ASCII}\\[3ex]
\hline
\textbf{...} & \textbf{...} & \multicolumn{16}{|c|}{...} \\[3ex]
\hline
\caption{Struktura tablicy atomów w pliku BEAM}
\label{table:atomTable} \\
\end{longtable}

%---------------------------------------------------------------------------
\subsection{Kod pośredni}
%---------------------------------------------------------------------------
Sekcja z kodem pośrednim zawiera faktyczny kod wykonywalny modułu, który jest interpretowany przez maszynę wirtualną w trakcie uruchomienia systemu.

Fragment pliku z kodem identyfikowana jest przez napis \texttt{Code}. Struktura danych fragmentu zawarta została w tabeli \ref{table:bytecode}. Szczegółowy opis reprezentacji i znaczenia opkodów i ich argumentów zawarty został w dodatku \ref{cha:operacjeBeam}.

\begin{longtable}{|c|c|c|c|c|c|c|c|c|c|c|c|c|c|c|c|c|c|}
\hline
         & \textbf{Oktet} & \multicolumn{8}{|c|}{\textbf{0}} & \multicolumn{8}{|c|}{\textbf{1}} \\
\hline
\textbf{Oktet} & \textbf{Bit} & \textbf{0} & \textbf{1} & \textbf{2} & \textbf{3} & \textbf{4} & \textbf{5} & \textbf{6} & \textbf{7} & \textbf{8} & \textbf{9} & \textbf{10} & \textbf{11} & \textbf{12} & \textbf{13} & \textbf{14} & \textbf{15}\\
\hline
\textbf{0} & \textbf{0} & \multicolumn{16}{|c|}{0x000010} \\[3ex]
\hline
\textbf{4} & \textbf{32} & \multicolumn{16}{|c|}{Numer wersji formatu kod (w Erlangu R16 - 0x00000000)}\\[3ex]
\hline
\textbf{8} & \textbf{64} & \multicolumn{16}{|c|}{Maksymalny numer operacji (do sprawdzenia kompatybilności)} \\[3ex]
\hline
\textbf{12} & \textbf{96} & \multicolumn{16}{|c|}{Liczba etykiet w kodzie modułu}\\[3ex]
\hline
\textbf{16} & \textbf{128} & \multicolumn{16}{|c|}{Liczba funkcji eksportowanych z modułu} \\[3ex]
\hline
\textbf{20} & \textbf{160} & \multicolumn{8}{|c|}{Opkod 1} & \multicolumn{8}{|c|}{Argument 1}  \\[3ex]
\hline
\textbf{...} & \textbf{...} & \multicolumn{8}{|c|}{...} & \multicolumn{8}{|c|}{Argument N}  \\[3ex]
\hline
\textbf{...} & \textbf{...} & \multicolumn{8}{|c|}{Opkod 2} & \multicolumn{8}{|c|}{Argument 1}  \\[3ex]
\hline
\textbf{...} & \textbf{...} & \multicolumn{16}{|c|}{...}  \\[3ex]
\hline
\caption{Struktura kodu pośredniego w pliku BEAM}
\label{table:bytecode} \\
\end{longtable}


%---------------------------------------------------------------------------
\subsection{Tablica importowanych funkcji}
%---------------------------------------------------------------------------
Fragment pliku binarnego z tablicą importowanych funkcji zawiera informacje o funkcjach zaimplementowanych w innych modułach, które są wykorzystywane przez moduł.

Identyfikowany jest on przez napis \texttt{ImpT}. Struktura danych fragmentu zawarta została w tabeli \ref{table:importtable}.

\begin{longtable}{|c|c|c|c|c|c|c|c|c|c|c|c|c|c|c|c|c|c|}
\hline
         & \textbf{Oktet} & \multicolumn{8}{|c|}{\textbf{0}} & \multicolumn{8}{|c|}{\textbf{1}} \\
\hline
\textbf{Oktet} & \textbf{Bit} & \textbf{0} & \textbf{1} & \textbf{2} & \textbf{3} & \textbf{4} & \textbf{5} & \textbf{6} & \textbf{7} & \textbf{8} & \textbf{9} & \textbf{10} & \textbf{11} & \textbf{12} & \textbf{13} & \textbf{14} & \textbf{15}\\
\hline
\textbf{0} & \textbf{0} & \multicolumn{16}{|c|}{Liczba importowanych funkcji} \\[3ex]
\hline
\textbf{4} & \textbf{32} & \multicolumn{16}{|c|}{Indeks atomu z nazwą modułu 1}\\[3ex]
\hline
\textbf{8} & \textbf{64} & \multicolumn{16}{|c|}{Indeks atomu z nazwą funkcji 1} \\[3ex]
\hline
\textbf{12} & \textbf{96} & \multicolumn{16}{|c|}{Arność funkcji 1}\\[3ex]
\hline
\textbf{16} & \textbf{128} & \multicolumn{16}{|c|}{Indeks atomu z nazwą modułu 2}\\[3ex]
\hline
\textbf{...} & \textbf{...} & \multicolumn{16}{|c|}{...}  \\[3ex]
\hline
\caption{Struktura tablicy importowanych funkcji w pliku BEAM}
\label{table:importtable} \\
\end{longtable}

%---------------------------------------------------------------------------
\subsection{Tablica eksportowanych funkcji}
%---------------------------------------------------------------------------
Fragment pliku binarnego z tablicą eksportowanych funkcji zawiera informacje o funkcjach z modułu, które widoczne są z poziomu innych modułów.

Identyfikowany jest on przez napis \texttt{ExpT}. Struktura danych fragmentu zawarta została w tabeli \ref{table:exporttable}.

\begin{longtable}{|c|c|c|c|c|c|c|c|c|c|c|c|c|c|c|c|c|c|}
\hline
         & \textbf{Oktet} & \multicolumn{8}{|c|}{\textbf{0}} & \multicolumn{8}{|c|}{\textbf{1}} \\
\hline
\textbf{Oktet} & \textbf{Bit} & \textbf{0} & \textbf{1} & \textbf{2} & \textbf{3} & \textbf{4} & \textbf{5} & \textbf{6} & \textbf{7} & \textbf{8} & \textbf{9} & \textbf{10} & \textbf{11} & \textbf{12} & \textbf{13} & \textbf{14} & \textbf{15}\\
\hline
\textbf{0} & \textbf{0} & \multicolumn{16}{|c|}{Liczba eksportowanych funkcji} \\[3ex]
\hline
\textbf{4} & \textbf{32} & \multicolumn{16}{|c|}{Indeks atomu z nazwą funkcji 1}\\[3ex]
\hline
\textbf{8} & \textbf{64} & \multicolumn{16}{|c|}{Arność funkcji 1} \\[3ex]
\hline
\textbf{12} & \textbf{96} & \multicolumn{16}{|c|}{Etykieta początku kodu funkcji 1}\\[3ex]
\hline
\textbf{16} & \textbf{128} & \multicolumn{16}{|c|}{Indeks atomu z nazwą funkcji 2}\\[3ex]
\hline
\textbf{...} & \textbf{...} & \multicolumn{16}{|c|}{...}  \\[3ex]
\hline
\caption{Struktura tablicy eksportowanych funkcji w pliku BEAM}
\label{table:exporttable} \\
\end{longtable}

%---------------------------------------------------------------------------
\subsection{Tablica funkcji lokalnych}
%---------------------------------------------------------------------------
Fragment pliku binarnego z tablicą lokalnych funkcji zawiera informacje o funkcjach zaimplementowanych w module (w tym lambd), które wykorzystywane są tylko przez ten moduł i nie są widoczne z poziomu innych modułów.

Identyfikowany jest on przez napis \texttt{LocT}. Struktura danych fragmentu zawarta została w tabeli \ref{table:localtable}.

\begin{longtable}{|c|c|c|c|c|c|c|c|c|c|c|c|c|c|c|c|c|c|}
\hline
         & \textbf{Oktet} & \multicolumn{8}{|c|}{\textbf{0}} & \multicolumn{8}{|c|}{\textbf{1}} \\
\hline
\textbf{Oktet} & \textbf{Bit} & \textbf{0} & \textbf{1} & \textbf{2} & \textbf{3} & \textbf{4} & \textbf{5} & \textbf{6} & \textbf{7} & \textbf{8} & \textbf{9} & \textbf{10} & \textbf{11} & \textbf{12} & \textbf{13} & \textbf{14} & \textbf{15}\\
\hline
\textbf{0} & \textbf{0} & \multicolumn{16}{|c|}{Liczba lokalnych funkcji} \\[3ex]
\hline
\textbf{4} & \textbf{32} & \multicolumn{16}{|c|}{Indeks atomu z nazwą funkcji 1}\\[3ex]
\hline
\textbf{8} & \textbf{64} & \multicolumn{16}{|c|}{Arność funkcji 1} \\[3ex]
\hline
\textbf{12} & \textbf{96} & \multicolumn{16}{|c|}{Etykieta początku kodu funkcji 1}\\[3ex]
\hline
\textbf{16} & \textbf{128} & \multicolumn{16}{|c|}{Indeks atomu z nazwą funkcji 2}\\[3ex]
\hline
\textbf{...} & \textbf{...} & \multicolumn{16}{|c|}{...}  \\[3ex]
\hline
\caption{Struktura tablicy lokalnych funkcji w pliku BEAM}
\label{table:localtable} \\
\end{longtable}

%---------------------------------------------------------------------------
\subsection{Tablica lambd}
%---------------------------------------------------------------------------
Fragment pliku binarnego z tablicą lambd zawiera informacje o obiektach funkcyjnych, które wykorzystywane są przez ten moduł.

Lambdy indentyfikowane są poprzez atomy, które powstały przez złączenie nazwy funkcji, w której zostały zdefiniowane oraz kolejny indeks lambdy zdefiniowanej w danej funkcji.
Np. kolejne obiekty funkcyjne zdefiniowane w funkcji \texttt{foo/1} będą identyfikowane przez atomy \texttt{-foo/1-fun-0-}, \texttt{-foo/1-fun-1-} itd.

Fragment pliku tablicą lambdy identyfikowany jest przez napis \texttt{FunT}. Struktura danych fragmentu zawarta została w tabeli \ref{table:lambdatable}.

\begin{longtable}{|c|c|c|c|c|c|c|c|c|c|c|c|c|c|c|c|c|c|}
\hline
         & \textbf{Oktet} & \multicolumn{8}{|c|}{\textbf{0}} & \multicolumn{8}{|c|}{\textbf{1}} \\
\hline
\textbf{Oktet} & \textbf{Bit} & \textbf{0} & \textbf{1} & \textbf{2} & \textbf{3} & \textbf{4} & \textbf{5} & \textbf{6} & \textbf{7} & \textbf{8} & \textbf{9} & \textbf{10} & \textbf{11} & \textbf{12} & \textbf{13} & \textbf{14} & \textbf{15}\\
\hline
\textbf{0} & \textbf{0} & \multicolumn{16}{|c|}{Liczba lambd w module} \\[3ex]
\hline
\textbf{4} & \textbf{32} & \multicolumn{16}{|c|}{Indeks atomu z identyfikatorem lambdy 1}\\[3ex]
\hline
\textbf{8} & \textbf{64} & \multicolumn{16}{|c|}{Arność lambdy 1} \\[3ex]
\hline
\textbf{12} & \textbf{96} & \multicolumn{16}{|c|}{Etykieta początku kodu lambdy 1}\\[3ex]
\hline
\textbf{16} & \textbf{128} & \multicolumn{16}{|c|}{Indeks lambdy 1 (0x00)}\\[3ex]
\hline
\textbf{20} & \textbf{160} & \multicolumn{16}{|c|}{Liczba wolnych zmiennych w lambdzie 1}\\[3ex]
\hline
\textbf{24} & \textbf{192} & \multicolumn{16}{|c|}{Wartość skrótu z drzewa syntaktycznego kodu lambdy 1}\\[3ex]
\hline
\textbf{28} & \textbf{224} & \multicolumn{16}{|c|}{Indeks atomu z identyfikatorem lambdy 2}\\[3ex]
\hline
\textbf{...} & \textbf{...} & \multicolumn{16}{|c|}{...}  \\[3ex]
\hline
\caption{Struktura tablicy lambd w pliku BEAM}
\label{table:lambdatable} \\
\end{longtable}

%---------------------------------------------------------------------------
\subsection{Tablica stałych}
%---------------------------------------------------------------------------
Fragment pliku binarnego z tablicą lambd stałych zawiera informacje o stałych (listy, napisy, duże liczby) które wykorzystywane są przez ten moduł.

Właściwa lista wartości stałych (od bajtu 4 do końca fragmentu) przechowywana jest w pliku w postaci skompresowanej algorytmem \textbf{zlib}.
Stałe zapisane są w formacie binarnym w formacie \emph{External Term Format}, opisanym w dokumencie \cite{ExternalTermFormat}.

Fragment pliku z tablicą identyfikowany jest przez napis \texttt{LitT}. Struktura danych fragmentu zawarta została w tabeli \ref{table:literaltable}.

\begin{longtable}{|c|c|c|c|c|c|c|c|c|c|c|c|c|c|c|c|c|c|}
\hline
         & \textbf{Oktet} & \multicolumn{8}{|c|}{\textbf{0}} & \multicolumn{8}{|c|}{\textbf{1}} \\
\hline
\textbf{Oktet} & \textbf{Bit} & \textbf{0} & \textbf{1} & \textbf{2} & \textbf{3} & \textbf{4} & \textbf{5} & \textbf{6} & \textbf{7} & \textbf{8} & \textbf{9} & \textbf{10} & \textbf{11} & \textbf{12} & \textbf{13} & \textbf{14} & \textbf{15}\\
\hline
\textbf{0} & \textbf{0} & \multicolumn{16}{|c|}{Rozmiar tablicy w bajtach} \\[3ex]
\hline
\textbf{4} & \textbf{32} & \multicolumn{16}{|c|}{Liczba stałych}\\[3ex]
\hline
\textbf{8} & \textbf{64} & \multicolumn{16}{|c|}{Rozmiar stałej 1 w bajtach} \\[3ex]
\hline
\textbf{12} & \textbf{96} & \multicolumn{16}{|c|}{Stała 1 w formacie binarnym}\\[6ex]
\hline
\textbf{...} & \textbf{...} & \multicolumn{16}{|c|}{Rozmiar stałej 2 w bajtach}\\[3ex]
\hline
\textbf{...} & \textbf{...} & \multicolumn{16}{|c|}{...}\\[6ex]
\hline
\caption{Struktura tablicy stałych w pliku BEAM}
\label{table:literaltable} \\
\end{longtable}

%---------------------------------------------------------------------------
\subsection{Lista atrybutów modułu}
%---------------------------------------------------------------------------
Fragment pliku binarnego z listą atrybutów modułu zawiera listę dwójek (proplistę) ze wszystkimi dodatkowymi atrybutami, z jakimi został skompilowany dany moduł (np. informacje o wersji czy autorze). Lista ta zapisana jest binarnie w postaci \emph{External Term Format}.

Fragment ten reprezentowany jest przez napis \texttt{Attr}.

%---------------------------------------------------------------------------
\subsection{Lista dodatkowych informacji o kompilacji modułu}
%---------------------------------------------------------------------------
Fragment pliku binarnego z listą informacji o kompilacji modułu zawiera proplistę z informacjami dotyczącymi kompilacji, takimi jak: ścieżka pliku z kodem źrodłowym, czas kompilacji, wersja kompilatora czy użyte opcje kompilacji. Informacje te zapisane są binarnie w postaci \emph{External Term Format}.

Fragment ten reprezentowany jest przez napis \texttt{CInf}.

%---------------------------------------------------------------------------
\subsection{Tablica linii kodu źródłowego modułu}
%---------------------------------------------------------------------------
Fragment pliku binarnego z informacjami o liniach kodu źródłowego modułu zawiera informacje dla instrukcji \texttt{line/1} maszyny wirtualnej o pliku źródłowym i linii, z której pochodzi aktualnie wykonywany fragment kodu. Informacje te wykorzystywane są przy generowaniu stosu wywołań przy wystąpięniu błędu lub wyjątku. Funkcjonalność ta została wprowadzona dopiero w wersji R15 maszyny wirtualnej Erlanga.

Jeżeli kompilowany plik jest na etapie preprocessingu łączony z innymi plikami z kodem źródłowym (przez użycie atrybutu \texttt{include}) to informacja o tych plikach zostanie zawarta w tym fragmencie. Domyślnie, kompilowany plik nie jest liczony i ma indeks 0. 

Numer linii koduje się przy użyciu tagu \texttt{0001}, jak w przypadku argumentów instrukcji maszyny wirtualnej, opisanych w sekcji \ref{sec:opsTypes}.
Rozróżnienie pliku, z którego pochodzi linia odbywa się za pomocą zapamiętania, z którego pliku pochodziła ostatnia linia. Domyślnie jest to plik o indeksie 0. Jeżeli dochodzi do zmiany aktualnego pliku, kolejny numer linii poprzedzony jest indeksem pliku z którego pochodzi, zakodowanym przy użyciu tagu \texttt{0010} (jak w sekcji \ref{sec:opsTypes}). Dlatego też numer linii może zawierać w pliku binarnym 1 lub 2 bajty.

Fragment pliku z tablicą identyfikowany jest przez napis \texttt{Line}. Struktura danych fragmentu zawarta została w tabeli \ref{table:linetable}.

\begin{longtable}{|c|c|c|c|c|c|c|c|c|c|c|c|c|c|c|c|c|c|}
\hline
         & \textbf{Oktet} & \multicolumn{8}{|c|}{\textbf{0}} & \multicolumn{8}{|c|}{\textbf{1}} \\
\hline
\textbf{Oktet} & \textbf{Bit} & \textbf{0} & \textbf{1} & \textbf{2} & \textbf{3} & \textbf{4} & \textbf{5} & \textbf{6} & \textbf{7} & \textbf{8} & \textbf{9} & \textbf{10} & \textbf{11} & \textbf{12} & \textbf{13} & \textbf{14} & \textbf{15}\\
\hline
\textbf{0} & \textbf{0} & \multicolumn{16}{|c|}{Wersja (0x000000)} \\[3ex]
\hline
\textbf{4} & \textbf{32} & \multicolumn{16}{|c|}{0x000000}\\[3ex]
\hline
\textbf{8} & \textbf{64} & \multicolumn{16}{|c|}{Liczba instrukcji \texttt{line} w kodzie modułu} \\[3ex]
\hline
\textbf{12} & \textbf{96} & \multicolumn{16}{|c|}{Liczba linii z kodem w plikach modułu}\\[3ex]
\hline
\textbf{16} & \textbf{128} & \multicolumn{16}{|c|}{Liczba plików z kodem modułu}\\[3ex]
\hline
\textbf{20} & \textbf{160} & \multicolumn{8}{|c|}{Numer linii (1 lub 2 B)} & \multicolumn{8}{|c|}{Numer linii (1 lub 2 B)} \\[3ex]
\hline
\textbf{...} & \textbf{...} & \multicolumn{16}{|c|}{...}\\[3ex]
\hline
\textbf{...} & \textbf{...} & \multicolumn{8}{|c|}{Długość nazwy pliku 1} & \multicolumn{8}{|c|}{Nazwa pliku 1 w ASCII}\\[3ex]
\hline
\textbf{...} & \textbf{...} & \multicolumn{16}{|c|}{...}\\[3ex]
\hline
\textbf{...} & \textbf{...} & \multicolumn{8}{|c|}{Długość nazwy pliku 2} & \multicolumn{8}{|c|}{Nazwa pliku 2 w ASCII}\\[3ex]
\hline
\textbf{...} & \textbf{...} & \multicolumn{16}{|c|}{...}\\[3ex]
\hline
\caption{Struktura tablicy linii kodu źródłowego w pliku BEAM}
\label{table:linetable} \\
\end{longtable}

%---------------------------------------------------------------------------
\subsection{Drzewo syntaktyczne modułu}
%---------------------------------------------------------------------------
Plik z modułem zawiera fragment pliku źródłowego z drzewem syntaktycznym pliku z kodem źródłowym o ile został skompilowany z opcją \texttt{debug\_info}.
Fragment ten identyfikowany jest przez napis \texttt{Abst}. 

Zawartością fragmentu jest drzewo syntaktyczne modułu, w takim formacie jak w sekcji \ref{sec:compilationSyntaxtree} w formacie \emph{External Term Format}.
