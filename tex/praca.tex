\documentclass[pdflatex,11pt]{aghdpl}
% \documentclass{aghdpl}               % przy kompilacji programem latex
% \documentclass[pdflatex,en]{aghdpl}  % praca w języku angielskim
\usepackage[polish]{babel}
\usepackage[utf8]{inputenc}
\usepackage{longtable}
\usepackage{multirow}

% dodatkowe pakiety
\usepackage{enumerate}
\usepackage{listings}
\usepackage{xcolor}
\usepackage{url}
\usepackage{booktabs}
\usepackage{amsmath}
\lstloadlanguages{TeX}

\lstset{
  literate={ą}{{\k{a}}}1
           {ć}{{\'c}}1
           {ę}{{\k{e}}}1
           {ó}{{\'o}}1
           {ń}{{\'n}}1
           {ł}{{\l{}}}1
           {ś}{{\'s}}1
           {ź}{{\'z}}1
           {ż}{{\.z}}1
           {Ą}{{\k{A}}}1
           {Ć}{{\'C}}1
           {Ę}{{\k{E}}}1
           {Ó}{{\'O}}1
           {Ń}{{\'N}}1
           {Ł}{{\L{}}}1
           {Ś}{{\'S}}1
           {Ź}{{\'Z}}1
           {Ż}{{\.Z}}1
}
%---------------------------------------------------------------------------
% Style for Erlang listings

\lstdefinestyle{erlang}{
  belowcaptionskip=1\baselineskip,
  breaklines=true,
  frame=L,
  xleftmargin=\parindent,
  language=C,
  showstringspaces=false,
  basicstyle=\footnotesize\ttfamily,
  keywordstyle=\bfseries\color{green!40!black},
  commentstyle=\itshape\color{purple!40!black},
  morekeywords={when, define, module, export, include},
  identifierstyle=\color{blue},
  stringstyle=\color{orange},
  captionpos=b,
  numbers=left,
}
%---------------------------------------------------------------------------

\author{Rafał Studnicki}
\shortauthor{R. Studnicki}

\titlePL{Podstawowa funkcjonalność Erlanga dla systemu FreeRTOS}
\titleEN{Implementation of basic features of Erlang for FreeRTOS}

\shorttitlePL{Podstawowa funkcjonalność Erlanga dla systemu FreeRTOS}
\shorttitleEN{Implementation of basic features of Erlang for FreeRTOS}

\thesistypePL{Praca magisterska}
\thesistypeEN{Master of Science Thesis}

\supervisorPL{dr inż. Piotr Matyasik}
\supervisorEN{Piotr Matyasik Ph.D}

\date{2014}

\departmentPL{Katedra Informatyki Stosowanej}
\departmentEN{Department of Applied Computer Science}

\facultyPL{Wydział Elektrotechniki, Automatyki, Informatyki i Inżynierii Biomedycznej}
\facultyEN{Faculty of Electrical Engineering, Automatics, Computer Science and Engineering in Biomedicine}

\acknowledgements{}



\setlength{\cftsecnumwidth}{10mm}

%---------------------------------------------------------------------------

\begin{document}

\titlepages

\tableofcontents
\clearpage

%\chapter{Wprowadzenie}
\label{cha:wprowadzenie}

W rozdziale uwzględniono wstępne informacje dotyczące programowania urządzeń wbudowanych, a także opisano dotychczasowe wykorzystanie języków funkcyjnych w programowaniu takich urządzeń.
Opisano w nim także cele oraz zawartość niniejszej pracy.

%---------------------------------------------------------------------------

\section{Programowanie i zastosowanie systemów wbudowanych}
\label{sec:systemyWbudowane}

System wbudowany jest to system komputerowy będący zazwyczaj integralną częścią urządzenia zawierającego elementy sprzętowe i mechaniczne.
W przeciwieństwie do komputerów ogólne przeznaczenia, których celem jest realizacja różnego rodzaju zadań w zależności od potrzeb ich użytkowników, systemy wbudowane realizują tylko jedno, konkretne zadanie.

W obecnych czasach, gdy dąży się do tego, by coraz większa liczba urządzeń powszechnego użytku była "inteligentna" i mogła spełniać swoja zadania całkowicie niezależnie od człowieka, systemy wbudowane wykorzystywane w coraz większej mierze. Przykładami zastosowań systemów wbudowanych mogą być np.:
\begin{itemize}
\item telefony komórkowe;
\item centrale telefoniczne;
\item sterowniki do robotów mechanicznych;
\item sprzęt sterujący samolotami i rakietami;
\item układy sterujące pracą silnika samochodowego, komputery pokładowe;
\item systemy alarmowe, antywłamaniowe, przeciwpożarowe;
\item sprzęt medyczny;
\item sprzęt pomiarowy.
\end{itemize}

Systemy wbudowane najczęściej implementowane są w oparciu o mikrokontrolery, czyli scalone systemy mikroprocesorowe zawierające na jednym, zintegrowanym układzie scalonym oprócz mikroprocesora również pamięć RAM, programu, układy wejścia-wyjścia, układy licznikowe oraz kontrolery przerwań. Zintegrowanie wszystkich tych elementów na jednej płytce pozwala na redukcję rozmiaru i poboru mocy takiego układu.

Spośród architektur projektowania oprogramowania przeznaczonego do programowania urządzeń wbudowanych można wymienić:
\begin{enumerate}
\item kontrola programu w pętli \\
program kontrolowany jest w pojedynczej pętli, wewnątrz której podejmowane są decyzje o sterowaniu elementami sprzętowymi lub programowymi;
\item kontrola programu przez przerwania \\
konkretne zadania programu wywoływane są przez wewnętrzne (np. zegary) lub zewnętrzne (np. odbiór danych z portu szeregowego) przerwania.
Architektura ta często mieszana jest z wykonywaniem programu w pętli. W takim podejściu zadania o wysokim priorytecie wywoływane są przez przerwania, natomiast zadania o niskim priorytecie wykonywane są w pętli;
\item wielozadaniowość z wywłaszczaniem \\
w tego typu systemach pomiędzy kodem programu a mikrokontrolerem znajduje się niskopoziomowe oprogramowanie (jądro) odpowiadające za przydzielanie czasu procesora dla wielu współbieżnych zadań, które mogą mieć różne priorytety wykonania. Planista (\emph{scheduler}) decyduje także w którym momencie powinno zostać obsłużone przerwanie;
\item wielozadaniowość bez wywłaszczania \\
tego rodzaju architektura jest bardzo podobna do wielozadaniowości z wywłaszczaniem, jednak jądro nie dokonuje samodzielnych decyzji o przerwaniu wykonywania któregoś ze współbieżnych zadań lecz pozostawia tę decyzję programiście;
\item mikrojądro \\
jest rozszerzeniem systemów obsługujących wielozadaniowość bez wywłaszczania lub z wywłaszczaniem poprzez dodanie np. zarządzania pamięcią, mechanizmów synchronizacji czy komunikacji pomiędzy współbieżnymi zadaniami do funkcjonalności jądra. Przykładami mikrojąder mogą być np. FreeRTOS, Enea OSE czy RTEMS;
\item jądro monolityczne
do funkcjonalności jądra dodaje funkcjonalności zapewniające komplet komunikacji z peryferiami systemu dodające do funkcjonalności np. system plików, stos TCP/IP do komunikacji sieciowej, czy sterowniki obsługi urządzeń zewnętrznych.
Spośród systemów z jądrami monolitycznymi można wymienić takie systemy jak Embedded Linux czy Windows CE.
\end{enumerate}

Można zauważyć, że wymienione architektury zostały uporządkowane względem złożoności projektowanego systemu, ale także pod względem złożoności występującego elementu pośredniego pomiędzy programowanym fizycznym urządzeniem a oprogramowaniem. Wraz ze wzrostem złożoności systemu rosną również wymagania sprzętowe konieczne do uruchomienia danego systemu, maleje jednak bezpośredni poziom kontroli programisty nad realizacją wymagań czasu rzeczywistego.
W niniejszej pracy rozważane będą sposoby implementacji oprogramowania na systemy wbudowane w oparciu o mikrojądra.

%---------------------------------------------------------------------------

\section{Wykorzystanie Erlanga w programowaniu systemów wbudowanych}
\label{sec:jezykiFunkcyjne}

Jim Gray w pracy \emph{Why Do Computers Stop And What Can Be Done About It?} \cite{Gray85whydo} na podstawie obserwacji procesu projektowania i budowy sprzętu wchodzącego w skład systemów komputerowych sformułował pewne postulaty dotyczące implementacji oprogramowania odpornego na błędy.
Były one następujące:
\begin{enumerate}
\item oprogramowanie powinno być modularne, co powinno zostać zapewnione przez wyabstrahowanie logiki w procesach. Komunikacja między procesami powinna zostać zapewniona przez mechanizm przesyłania wiadomości;
\item propagacja błędów powinna być powstrzymywana tak szybko jak to tylko możliwe (\emph{fail-fast});
\item logika wykonywana przez procesy powinna być zduplikowana w całym systemie tak, aby możliwe było jej wykonanie pomimo błędu sprzętowego lub tymczasowego błędu innego modułu;
\item powinien zostać zapewniony mechanizm transakcyjny pozwalający na zachowanie spójności danych;
\item powinien zostać zapewniony mechanizm transakcyjny, który w połączeniu z duplikacją procesów ułatwi obsługę wyjątków i tolerowanie błędów oprogramowania.
\end{enumerate}

Obserwacje te były motywacją dla czwórki inżynierów z firmy Ericsson AB - Bjarne Dackera, Joe Armstrona, Mike'a Williams i Roberta Virdinga do stworzenia nowej platformy spełniającej powyższe wymagania i w oparciu o którą możliwe byłoby tworzenie projektów wewnątrz firmy.
Jak się później okazało, do zaspokojenia wszystkich wymienionych potrzeb konieczne było stworzenie nowego, dedykowanego języka programowania - Erlang, wraz z zestawem bibliotek - OTP (Open Telecom Platform).

Rozwiązanie to zapewniało realizację powyższych postulatów poprzez następujące cechy charakterystyczne:
\begin{enumerate}
\item izolowane, lekkie i możliwe do szybkiego uruchomienia procesy, które nie mogą bezpośrednio oddziaływać na inne uruchomione w systemie;
\item współbieżne uruchomienie procesów;
\item możliwość wykrywania błędów w jednym procesie przez drugi (monitorowanie procesów);
\item możliwość zidentyfikowania błędu i podjęcia odpowiedniej akcji w jego efekcie;
\item możliwość podmiany kodu uruchamianego programu w locie;
\item niezawodna baza danych (mnesia wchodząca w skład OTP).
\end{enumerate}

Należy zaznaczyć, że celem przyświacającym twórcom języka od samego początku było zastosowanie go w urządzeniach w budowanych, jak np. w centrali telekomunikacyjnej Ericsson AXD301, która pozostaje tego typu urządzeniem o największej liczbie sprzedanych egzemplarzy.

W tym miejscu nie można jednak zapomnieć o tym, że język ten został zaprojektowany dla systemów o miękkich wymaganiach czasu rzeczywistego. Systemy tego typu charakteryzują się tym, że oczekiwane czasy odpowiedzi w tym systemie są rzędu milisekund a odstępstwa od oczekiwanego czasu odpowiedzi powodują tylko spadek jakości usług danego systemu. W przeciwieństwie do tego, systemy o twardych wymaganiach czasu rzeczywistego uznaje się w takich sytuacjach za niefunkcjonujące.


Dystrybucja maszyny wirtualnej Erlanga BEAM (Bjorn/Bogdan Erlang Abstract Machine), która utrzymywana jest przez firmę Ericsson AB umożliwia uruchomienie jej w trybie wbudowanym na takich systemach operacyjnych jak VxWorks czy Embedded Solaris. Pierwszy z nich jest systemem operacyjnym czasu rzeczywistego, jednak maszyna wirtualna została przeniesiona na ten system tylko w zakresie pozwalającym na uruchomienie na niej centrali telekomunikacyjnej, a jej uruchomienie wymaga 32 MB pamięci RAM i 22 MB przestrzeni dyskowej.
Z kolei uruchomienie Erlang/OTP na systemie Embedded Solaris wymaga 17 MB pamięci RAM i 80 MB przestrzeni dyskowej.
Szczegóły dotyczące wersji maszyny wirtualnej na te systemy operacyjne mogą zostać znalezione w dokumentacji Erlang/OTP \cite{ErlangVxWorks}.


Oprócz tego, aktualnie rozwijanym, otwartym projektem związanym z uruchomieniem Erlanga na systemach wbudowanych jest Embedded Erlang. Powstający on przy współpracy firmy Erlang Solutions Ltd. Skupia się on jednak na uruchomieniu maszyny wirtualnej na Raspberry Pi oraz Paralleli, które wymagają pełnej dystrybucji systemu operacyjnego Linux. Szczegóły dotyczące projektu można znaleźć na stronie http://www.erlang-embedded.com.


Zatem wymagania, jakich potrzebują zarówno wymienione przeniesienia (\emph{porty}) maszyny BEAM oraz Embedded Erlang są zdecydowanie zbyt wysokie w porównaniu do specyfikacji sprzętowych rozważanych w niniejszej pracy.


W momencie powstawania pracy firma Ericsson AB była w trakcie implementacji maszyny wirtualnej Erlanga dla systemu operacyjnego czasu rzeczywistego Enea OSE. System ten abstrahuje logikę implementowanego oprogramowania w izolowanych procesach, komunikujących się między sobą poprzez wiadomości (\emph{actor model}). Poziom zgodności z filozofią Erlanga sprawia, że OSE wydaje się być idealnym systemem do implementacji maszyny wirtualnej dla tego języka. Pozostaje on jednak produktem zamkniętym.

Innym projektem godnym uwagi jest Grisp, autorstwa Peera Stritzingera, będący portem maszyny wirtualnej Erlanga dla mikrojądra RTEMS \cite{Stritzinger2013}. W momencie pisania pracy Grisp również pozostaje w trakcie rozwoju, jednak tak jak i maszyna dla systemu OSE pozostaje projektem zamkniętym.
%---------------------------------------------------------------------------

\section{Cele pracy}
\label{sec:celePracy}

Oczekiwanym efektem niniejszej pracy jest implementacja funkcjonalności systemu uruchomieniowego dla funkcyjnego, współbieżnego języka programowania Erlang dla systemu operacyjnego czasu rzeczywistego FreeRTOS.
Zakres implementacji powinien pozwolić na uruchomienie kodu pośredniego (bajtkodu) maszyny wirtualnej Erlanga skompilowanego przez kompilator maszyny wirtualnej BEAM na mikrokontrolerach o ograniczonych zasobach sprzętowych (jak np. mikrokontroler z serii LPC17xx, mający 512kB pamięci flash i 64kB pamięci RAM).

Sposób implementacji powinien pozwolić na uruchamianie programów w taki sposób, by możliwe było spełnienie przynajmniej pierwszych czterech cech charakterystycznych dla języka Erlang z podrozdziału \ref{sec:jezykiFunkcyjne}. Punkt 6. został na tym etapie pominięty, gdyż integracja systemu FreeRTOS z obsługą systemu plików leży poza zakresem pracy. Udostępnienie interfejsów sieciowych oraz możliwość korzystania z mechanizmu klastra Erlanga (\emph{Distributed Erlang}) również nie jest jednym z celów niniejszej pracy.

Celem pracy jest zatem umożliwienie implementacji oprogramowania uruchamianego w ramach systemach wbudowanych o miękkich wymaganiach czasu rzeczywistego, przy pomocy języka programowania Erlang. 

%---------------------------------------------------------------------------

\section{Zawartość pracy}
\label{sec:zawartoscPracy}
%\chapter{System operacyjny FreeRTOS}
\label{cha:freertos}

W niniejszym rozdziale opisano architekturę systemu (mikrojądra) FreeRTOS wraz ze sposobem, w jaki
poszczególne funkcjonalności mogą być przydatne w implementacji maszyny wirtualnej Erlanga dedykowanej
dla tego systemu.

%---------------------------------------------------------------------------
\section{Wprowadzenie}
\label{sec:rtosWprowadzenie}

Podstawową częścią każdego systemu operacyjnego jest jego jądro
FreeRTOS jest mikrojądrem, umożliwiającym tworzenie aplikacji czasu rzeczywistego (zarówno o miękkich jak i twardych wymaganiach) przeznaczonych na systemy wbudowane. 

%---------------------------------------------------------------------------
\section{Zadania i planista (\emph{scheduler})}
\label{sec:rtosScheduler}

%---------------------------------------------------------------------------
\section{Kolejki}
\label{sec:rtosKolejki}

%---------------------------------------------------------------------------
\section{Przerwania}
\label{sec:rtosPrzerwania}

%---------------------------------------------------------------------------
\section{Zarządzanie zasobami}
\label{sec:rtosZasoby}

%---------------------------------------------------------------------------
\section{Zarządzanie pamięcią}
\label{sec:rtosPamiec}

%---------------------------------------------------------------------------
\section{FreeRTOS i LPC17xx}
\label{sec:rtosLPC}

Mikrokontroler LPC17xx jest jednym z systemów, na który przeniesiony został mikrojądrem FreeRTOS.

%---------------------------------------------------------------------------
\section{Podsumowanie}
\label{sec:rtosPodsumowanie}
%\chapter{Język programowania Erlang}
\label{cha:erlang}

%---------------------------------------------------------------------------
\section{Wprowadzenie}
\label{sec:erlangWprowadzenie}
%---------------------------------------------------------------------------
\section{System typów danych}
\label{sec:erlangTypy}

Tu będzie coś o typach danych w Erlangu i ich reprezentacji w pamięci.
%---------------------------------------------------------------------------
%---------------------------------------------------------------------------

\appendix
\chapter{Kompilacja kodu źródłowego}
\label{cha:erlangKompilacja}

Dodatek opisuje kolejne kroki, z jakich składa się proces otrzymywania skompilowanego kodu pośredniego maszyny wirtualnej BEAM z kodu źródłowego napisanego w języku Erlang.
Oprócz tego dodatek dokumentuje, na potrzeby projektu, zawartość pliku ze skompilowanym kodem pośrednim.  Format pliku nie jest objęty oficjalną dokumentacją języka ze względu na dużą zmienność pomiędzy kolejnymi wersjami kompilatora i maszyny wirtualnej.

%---------------------------------------------------------------------------
\section{Wprowadzenie}

Narzędzia przeznaczone do generacji wszystkich form pośrednich kodu źródłowego opisanych w~niniejszym rozdziale zostały napisane w języku Erlang. Dostępne są one w pakiecie aplikacji \texttt{compiler} dostarczanej wraz z maszyną wirtualną BEAM.

%---------------------------------------------------------------------------
\section{Kod źródłowy}

Listingi \ref{lis:facERL} oraz \ref{lis:facHRL} prezentują prosty, przykładowy moduł zaimplementowany w języku Erlang, którego zadaniem jest obliczanie silni za pomocą funkcji ogonowo-rekurencyjnej.

Elementy takie jak używanie rekordu do przechowywania akumulatora funkcji czy zwracanie krotki informującej o błędzie zaciemniają nieco sposób działania funkcji. W tym przykładzie zostały one jednak zawarte w celu przeanalizowania przekształceń, jakim poddawany jest kod źródłowy w~trakcie poszczególnych kroków kompilacji.

\begin{lstlisting}[style=erlang, caption=Plik fac.erl, label=lis:facERL]
-module(fac).

-export([fac/1]).
-define(ERROR, "Invalid argument").

-include("fac.hrl").

fac(#factorial{n=0, acc=Acc}) ->
    Acc;
fac(#factorial{n=N, acc=Acc}) ->
    fac(#factorial{n=N-1, acc=N*Acc});
fac(N) when is_integer(N) ->
    fac(#factorial{n=N});
fac(N) when is_binary(N) ->
    fac(binary_to_integer(N));
fac(_) ->
    {error, ?ERROR}.
\end{lstlisting}

\begin{lstlisting}[style=erlang, caption=Plik fac.hrl, label=lis:facHRL]
-record(factorial, {n, acc=1}).
\end{lstlisting}

%---------------------------------------------------------------------------
\section{Preprocessing}

Pierwszym krokiem w procesie otrzymywania kodu maszynowego Erlanga jest wstępne przetwarzanie wykonywane przez preprocesor kompilatora.

Krok ten jest dwuetapowy. W pierwszym, z plikiem modułu łączone są pliki, które zostały do niego włączone poprzez użycie dyrektywy \texttt{\-include}. W miejscach dołączania zewnętrznych plików dodawane są także dyrektywy \texttt{\-file} po to, aby przy debugowaniu skompilowanego modułu można było zidentyfikować z którego pliku pochodzą dane fragmenty kodu. Na tym etapie podstawiane są również wartości makr. Jeżeli w opcjach kompilacji zdefiniowana jest operacja \texttt{\{parse\_transform, Module\}} modyfikująca drzewo składniowe modułu to zostanie ona również wykonywana w tym kroku, przekazując do dalszej kompilacji wynik działania funkcji \texttt{Module:parse\_transform/2}.

Kod modułu po tym kroku został przedstawiony na listingu \ref{lis:facP}. Wygenerowanie kodu w tej postaci jest możliwe przy użyciu kompilatora z opcją \texttt{-P}: \texttt{erlc -P fac.erl}.

Kolejnym krokiem jest rozwinięcie rekordów do krotek (rekordy są tylko lukrem składniowym języka Erlang, upraszczającym operacje na krotkach o dużej liczbie pól) oraz dołączenie funkcji \texttt{module\_info/0} oraz \texttt{module\_info/1}, które znajdują się w każdym skompilowanym module.
Funkcje te zwracają informacje o skompilowanym module, takie jak np. eksportowane funkcje.

Kod modułu po wykonaniu tej operacji został przedstawiony na listingu \ref{lis:facE}. Wygenerowanie kodu w tej postaci jest możliwe przy użyciu kompilatora z opcją \texttt{-E}: \texttt{erlc -E fac.erl}.

\begin{lstlisting}[style=erlang, caption=Moduł \texttt{fac} po pierwszym przetworzeniu, label=lis:facP]
-file("fac.erl", 1).

-module(fac).

-export([fac/1]).

-file("fac.hrl", 1).

-record(factorial,{n,acc = 1}).

-file("fac.erl", 7).

fac(#factorial{n = 0,acc = Acc}) ->
    Acc;
fac(#factorial{n = N,acc = Acc}) ->
    fac(#factorial{n = N - 1,acc = N * Acc});
fac(N) when is_integer(N) ->
    fac(#factorial{n = N});
fac(N) when is_binary(N) ->
    fac(binary_to_integer(N));
fac(_) ->
    {error,"Invalid argument"}.
\end{lstlisting}

\begin{lstlisting}[style=erlang, caption=Moduł \texttt{fac} po drugim przetworzeniu, label=lis:facE]
-file("fac.erl", 1).

-file("fac.hrl", 1).

-file("fac.erl", 7).

fac({factorial,0,Acc}) ->
    Acc;
fac({factorial,N,Acc}) ->
    fac({factorial,N - 1,N * Acc});
fac(N) when is_integer(N) ->
    fac({factorial,N,1});
fac(N) when is_binary(N) ->
    fac(binary_to_integer(N));
fac(_) ->
    {error,"Invalid argument"}.

module_info() ->
    erlang:get_module_info(fac).

module_info(X) ->
    erlang:get_module_info(fac, X).
\end{lstlisting}
%---------------------------------------------------------------------------
\section{Drzewo składniowe}
\label{sec:compilationSyntaxtree}

Formatem, jakiego używa kompilator Erlanga do wykonywania poszczególnych kroków kompilacji jest drzewo składniowe modułu. Warto w tym miejscu przypomnieć, że programista również może ingerować w drzewo składniowe modułu, używając wspomnianej już opcji kompilacji \texttt{parse\_transform}.

Drzewo składniowe dla przykładowego modułu \texttt{fac}, po przetwarzaniu wstępnym, zostało przedstawione na listingu \ref{lis:facAE}.

\begin{lstlisting}[style=erlang, caption=Drzewo składniowe modułu fac, label=lis:facAE]
[{attribute,1,file,{"fac.erl",1}},
         {attribute,1,file,{"fac.hrl",1}},
         {attribute,7,file,{"fac.erl",7}},
         {function,8,fac,1,
             [{clause,8,
                  [{tuple,8,[{atom,8,factorial},{integer,8,0},{var,8,'Acc'}]}],
                  [],
                  [{var,9,'Acc'}]},
              {clause,10,
                  [{tuple,10,
                       [{atom,10,factorial},{var,10,'N'},{var,10,'Acc'}]}],
                  [],
                  [{call,11,
                       {atom,11,fac},
                       [{tuple,11,
                            [{atom,11,factorial},
                             {op,11,'-',{var,11,'N'},{integer,11,1}},
                             {op,11,'*',{var,11,'N'},{var,11,'Acc'}}]}]}]},
              {clause,12,
                  [{var,12,'N'}],
                  [[{call,12,
                        {remote,12,{atom,12,erlang},{atom,12,is_integer}},
                        [{var,12,'N'}]}]],
                  [{call,13,
                       {atom,13,fac},
                       [{tuple,13,
                            [{atom,13,factorial},
                             {var,13,'N'},
                             {integer,1,1}]}]}]},
              {clause,14,
                  [{var,14,'N'}],
                  [[{call,14,
                        {remote,14,{atom,14,erlang},{atom,14,is_binary}},
                        [{var,14,'N'}]}]],
                  [{call,15,
                       {atom,15,fac},
                       [{call,15,
                            {remote,15,
                                {atom,15,erlang},
                                {atom,15,binary_to_integer}},
                            [{var,15,'N'}]}]}]},
              {clause,16,
                  [{var,16,'_'}],
                  [],
                  [{tuple,17,
                       [{atom,17,error},{string,17,"Invalid argument"}]}]}]},
         {function,0,module_info,0,
             [{clause,0,[],[],
                  [{call,0,
                       {remote,0,{atom,0,erlang},{atom,0,get_module_info}},
                       [{atom,0,fac}]}]}]},
         {function,0,module_info,1,
             [{clause,0,
                  [{var,0,'X'}],
                  [],
                  [{call,0,
                       {remote,0,{atom,0,erlang},{atom,0,get_module_info}},
                       [{atom,0,fac},{var,0,'X'}]}]}]}]
\end{lstlisting}

Drzewo składniowe modułów może zostać także wykorzystane w sytuacji, jeżeli chcemy utworzyć kompilator innego języka programowania, który kompilowałby kod źródłowy w tym języku do kodu maszynowego Erlanga. W efekcie możliwe będzie uruchamianie programów napisanych w tym języku na dowolnej maszynie wirtualnej Erlanga. Wydaje się to być dobrym pomysłem dla języków, które mogą wynieść dużo korzyści z uruchamiania programów w nich napisanych na maszynie mającej takie właściwości jak BEAM. Przykładem tego może być język Elixir \cite{elixir}.
%---------------------------------------------------------------------------
\section{Core Erlang}
\label{sec:compilationCore}

Kolejnym krokiem kompilacji jest transformacja kodu do innego języka - Core Erlang. Jest to język funkcyjny, składnią przypominający język Erlang. Jednak ze względu na uproszczenie składni, pozwala on na łatwiejszą maszynową optymalizację i konwersję do kodu pośredniego maszyny wirtualnej (bajtkodu).

Kod rozważanego w tym dodatku modułu, w języku Core Erlang, został umieszczony na listingu \ref{lis:facCORE}.

\begin{lstlisting}[style=erlang, caption=Moduł \texttt{fac} w Core Erlang, label=lis:facCORE]
module 'fac' ['fac'/1,
	      'module_info'/0,
	      'module_info'/1]
    attributes []
'fac'/1 =
    %% Line 4
    fun (_cor0) ->
	case _cor0 of
	  <1> when 'true' ->
	      %% Line 5
	      1
	  %% Line 6
	  <N>
	      when call 'erlang':'is_integer'
		    (_cor0) ->
	      let <_cor1> =
		  %% Line 7
		  call 'erlang':'-'
		      (N, 1)
	      in  let <_cor2> =
		      %% Line 7
		      apply 'fac'/1
			  (_cor1)
		  in  %% Line 7
		      call 'erlang':'*'
			  (N, _cor2)
	  %% Line 8
	  <_X_Other> when 'true' ->
	      %% Line 9
	      'not_integer'
	end
'module_info'/0 =
    fun () ->
	call 'erlang':'get_module_info'
	    ('fac')
'module_info'/1 =
    fun (_cor0) ->
	call 'erlang':'get_module_info'
	    ('fac', _cor0)
end
\end{lstlisting}

%---------------------------------------------------------------------------
\section{Kod pośredni -  bajtkod}

Dopiero z modułu w postaci Core Erlang generowany jest bajtkod - kod maszynowy rozumiany przez maszynę wirtualną Erlanga. Język maszynowy zawiera instrukcje z określonego zestawu instrukcji, których pełna lista wraz z opisem argumentów znajduje się w dodatku \ref{cha:operacjeBeam}.

Wygenerowanie kodu w tej postaci jest możliwe przy użyciu kompilatora z opcją \texttt{-S}: \texttt{erlc -S fac.erl}.

Wygenerowany kod pośredni dla przykładowego modułu silni, w formie listy krotek, został zawarty na listingu \ref{facS}. Na listingu zawarty jest kod pośredni dla każdej z funkcji modułu (oznaczonej krotką \texttt{\{function, ...\}}). Pierwszym elementem każdej krotki wewnątrz funkcji jest nazwa operacji, a~kolejnymi argumenty tej operacji.

\begin{lstlisting}[style=erlang, caption=\emph{Bytecode} modułu fac, label=facS]
{module, fac}.  %% version = 0

{exports, [{fac,1},{module_info,0},{module_info,1}]}.

{attributes, []}.

{labels, 11}.


{function, fac, 1, 2}.
  {label,1}.
    {line,[{location,"fac.erl",8}]}.
    {func_info,{atom,fac},{atom,fac},1}.
  {label,2}.
    {test,is_tuple,{f,4},[{x,0}]}.
    {test,test_arity,{f,4},[{x,0},3]}.
    {get_tuple_element,{x,0},0,{x,1}}.
    {get_tuple_element,{x,0},1,{x,2}}.
    {get_tuple_element,{x,0},2,{x,3}}.
    {test,is_eq_exact,{f,4},[{x,1},{atom,factorial}]}.
    {test,is_eq_exact,{f,3},[{x,2},{integer,0}]}.
    {move,{x,3},{x,0}}.
    return.
  {label,3}.
    {line,[{location,"fac.erl",11}]}.
    {gc_bif,'-',{f,0},4,[{x,2},{integer,1}],{x,0}}.
    {line,[{location,"fac.erl",11}]}.
    {gc_bif,'*',{f,0},4,[{x,2},{x,3}],{x,1}}.
    {test_heap,4,4}.
    {put_tuple,3,{x,2}}.
    {put,{atom,factorial}}.
    {put,{x,0}}.
    {put,{x,1}}.
    {move,{x,2},{x,0}}.
    {call_only,1,{f,2}}.
  {label,4}.
    {test,is_integer,{f,5},[{x,0}]}.
    {test_heap,4,1}.
    {put_tuple,3,{x,1}}.
    {put,{atom,factorial}}.
    {put,{x,0}}.
    {put,{integer,1}}.
    {move,{x,1},{x,0}}.
    {call_only,1,{f,2}}.
  {label,5}.
    {test,is_binary,{f,6},[{x,0}]}.
    {allocate,0,1}.
    {line,[{location,"fac.erl",15}]}.
    {call_ext,1,{extfunc,erlang,binary_to_integer,1}}.
    {call_last,1,{f,2},0}.
  {label,6}.
    {move,{literal,{error,"Invalid argument"}},{x,0}}.
    return.


{function, module_info, 0, 8}.
  {label,7}.
    {line,[]}.
    {func_info,{atom,fac},{atom,module_info},0}.
  {label,8}.
    {move,{atom,fac},{x,0}}.
    {line,[]}.
    {call_ext_only,1,{extfunc,erlang,get_module_info,1}}.


{function, module_info, 1, 10}.
  {label,9}.
    {line,[]}.
    {func_info,{atom,fac},{atom,module_info},1}.
  {label,10}.
    {move,{x,0},{x,1}}.
    {move,{atom,fac},{x,0}}.
    {line,[]}.
    {call_ext_only,2,{extfunc,erlang,get_module_info,2}}.
\end{lstlisting}

Kod modułu w postaci kodu pośredniego jest jeszcze na zbyt wysokim poziomie abstrakcji, aby mógł bezpośrednio zostać zrozumiany przez maszynę wirtualną. Wszystkie argumenty operacji użyte w~kodzie, takie jak np. odnośniki do etykiet czy atomy muszą zostać zapisane w odpowiednich tablicach i ich indeksy w odpowiedniej formie mogą dopiero zostać użyte jako właściwe argumenty operacji. Oprócz tego same nazwy operacji muszą zostać zamienione na odpowiadające im opkody. Czynności te wykonywane są w kolejnym kroku - generacji pliku binarnego z kodem modułu, opisanym w kolejnej sekcji.

%---------------------------------------------------------------------------
\section{Plik binarny BEAM}
%---------------------------------------------------------------------------

Efektem przetworzenia kodu pośredniego, wyrażonego w postaci krotek, jest plik binarny w formacie IFF \cite{morrison1985ea}, w formacie zrozumiałym przez maszynę wirtualną BEAM. Maszyna ta wykorzystuje tego rodzaju pliki do ładowania kodu poszczególnych modułów do pamięci. Ich źródłem może być zarówno system plików na fizycznej maszynie, na której uruchomiony został BEAM, jak i inna maszyna wirtualna znajdująca się w tym samym klastrze \emph{Distributed Erlang}, co docelowa.

W tabeli \ref{table:beamFile} zaprezentowana została ogólna struktura pliku binarnego ze skompilowanym modułem.

\begin{longtable}{|c|c|c|c|c|c|c|c|c|c|c|c|c|c|c|c|c|c|}
\hline
         & \textbf{Oktet} & \multicolumn{8}{|c|}{\textbf{0}} & \multicolumn{8}{|c|}{\textbf{1}} \\
\hline
\textbf{Oktet} & \textbf{Bit} & \textbf{0} & \textbf{1} & \textbf{2} & \textbf{3} & \textbf{4} & \textbf{5} & \textbf{6} & \textbf{7} & \textbf{8} & \textbf{9} & \textbf{10} & \textbf{11} & \textbf{12} & \textbf{13} & \textbf{14} & \textbf{15}\\
\hline
\textbf{0} & \textbf{0} & \multicolumn{16}{|c|}{"FOR1"} \\[3ex]
\hline
\textbf{4} & \textbf{32} & \multicolumn{16}{|c|}{Rozmiar pliku bez pierwszych 8 bajtów}\\[3ex]
\hline
\textbf{8} & \textbf{64} & \multicolumn{16}{|c|}{"BEAM"} \\[3ex]
\hline
\textbf{12} & \textbf{96} & \multicolumn{16}{|c|}{Identyfikator fragmentu (\emph{chunk}) 1}\\[3ex]
\hline
\textbf{16} & \textbf{128} & \multicolumn{16}{|c|}{Rozmiar fragmentu 1} \\[3ex]
\hline
\textbf{20} & \textbf{160} & \multicolumn{16}{|c|}{Dane fragmentu 1} \\[10ex]
\hline
\textbf{...} & \textbf{...} & \multicolumn{16}{|c|}{Identyfikator fragmentu (\emph{chunk}) 2}\\[3ex]
\hline
\textbf{...} & \textbf{...} & \multicolumn{16}{|c|}{...} \\[10ex]
\hline
\caption{Struktura pliku modułu BEAM}
\label{table:beamFile} \\
\end{longtable}

Każdy plik binarny BEAM powinien zawierać przynajmniej 4 następujące fragmenty (\emph{chunki}).
Obok opisu każdego fragmentu, w nawiasie podano ciąg znaków będący jego identyfikatorem w binarnym pliku modułu:
\begin{itemize}
\item tablica atomów wykorzystywanych przez moduł (\texttt{Atom});
\item kod pośredni danego modułu (\texttt{Code});
\item tablica zewnętrznych funkcji używanych przez moduł (\texttt{ImpT});
\item tablica funkcji eksportowanych przez moduł (\texttt{ExpT}).
\end{itemize}

Ponadto, w pliku mogą znajdować się następujące fragmenty:
\begin{itemize}
\item tablica funkcji lokalnych dla danego modułu (\texttt{LocT});
\item tablica lambd wykorzystywanych przed moduł (\texttt{FunT});
\item tablica stałych wykorzystywanych przed moduł (\texttt{LitT});
\item lista atrybutów modułu (\texttt{Attr});
\item lista dodatkowych informacji o kompilacji modułu (\texttt{CInf)};
\item tablica linii kodu źródłowego modułu (\texttt{Line});
\item drzewo syntaktyczne modułu (\texttt{Abst}).
\end{itemize}

W przypadku każdego rodzaju fragmentu, obszar pamięci jaki zajmuje on w pliku jest zawsze wielokrotnością 4 bajtów. Nawet jeżeli nagłówek fragmentu, zawierający jego rozmiar nie jest podzielny przez 4, obszar zaraz za danym fragmentem dopełniany jest zerami do pełnych 4 bajtów.

Warto zaznaczyć również, że sposób implementacji maszyny wirtualnej BEAM nie definiuje kolejności w jakiej poszczególne fragmenty powinny występować w pliku binarnym.

%---------------------------------------------------------------------------
\subsection{Tablica atomów}
%---------------------------------------------------------------------------
Tablica atomów zawiera listę wszystkich atomów, które używane są przez dany moduł. W trakcie ładowania kodu modułu przez maszynę wirtualną, atomy, które nie występowały we wcześniej załadowanych modułach, zostają wstawione do globalnej tablicy atomów (w postaci tablicy z hashowaniem).

Ponieważ długość atomu zapisana jest na jednym bajcie, nazwa atomu może mieć maksymalnie 255 znaków.

Fragment piku binarnego z tablicą atomów reprezentowany jest przez napis \texttt{Atom}. Struktura danych fragmentu zaprezentowana jest w tabeli \ref{table:atomTable}.

\begin{longtable}{|c|c|c|c|c|c|c|c|c|c|c|c|c|c|c|c|c|c|}
\hline
         & \textbf{Oktet} & \multicolumn{8}{|c|}{\textbf{0}} & \multicolumn{8}{|c|}{\textbf{1}} \\
\hline
\textbf{Oktet} & \textbf{Bit} & \textbf{0} & \textbf{1} & \textbf{2} & \textbf{3} & \textbf{4} & \textbf{5} & \textbf{6} & \textbf{7} & \textbf{8} & \textbf{9} & \textbf{10} & \textbf{11} & \textbf{12} & \textbf{13} & \textbf{14} & \textbf{15}\\
\hline
\textbf{0} & \textbf{0} & \multicolumn{16}{|c|}{Ilość atomów w tablicy atomów} \\[3ex]
\hline
\textbf{4} & \textbf{32} & \multicolumn{8}{|c|}{Dł. atomu 1} & \multicolumn{8}{|c|}{Nazwa atomu 1 w ASCII}\\[3ex]
\hline
\textbf{...} & \textbf{...} & \multicolumn{8}{|c|}{Dł. atomu 2} & \multicolumn{8}{|c|}{Nazwa atomu 2 w ASCII}\\[3ex]
\hline
\textbf{...} & \textbf{...} & \multicolumn{16}{|c|}{...} \\[3ex]
\hline
\caption{Struktura tablicy atomów w pliku BEAM}
\label{table:atomTable} \\
\end{longtable}

%---------------------------------------------------------------------------
\subsection{Kod pośredni}
%---------------------------------------------------------------------------
Sekcja z kodem pośrednim zawiera faktyczny kod wykonywalny modułu, który jest interpretowany przez maszynę wirtualną w trakcie uruchomienia systemu.
Szczegółowy opis reprezentacji i znaczenia opkodów i ich argumentów zawarty został w dodatku \ref{cha:operacjeBeam}.

Fragment pliku z kodem identyfikowana jest przez napis \texttt{Code}. Struktura danych fragmentu zawarta została w tabeli \ref{table:bytecode}. 

\begin{longtable}{|c|c|c|c|c|c|c|c|c|c|c|c|c|c|c|c|c|c|}
\hline
         & \textbf{Oktet} & \multicolumn{8}{|c|}{\textbf{0}} & \multicolumn{8}{|c|}{\textbf{1}} \\
\hline
\textbf{Oktet} & \textbf{Bit} & \textbf{0} & \textbf{1} & \textbf{2} & \textbf{3} & \textbf{4} & \textbf{5} & \textbf{6} & \textbf{7} & \textbf{8} & \textbf{9} & \textbf{10} & \textbf{11} & \textbf{12} & \textbf{13} & \textbf{14} & \textbf{15}\\
\hline
\textbf{0} & \textbf{0} & \multicolumn{16}{|c|}{0x000010} \\[3ex]
\hline
\textbf{4} & \textbf{32} & \multicolumn{16}{|c|}{Numer wersji formatu kod (w Erlangu R16 - 0x00000000)}\\[3ex]
\hline
\textbf{8} & \textbf{64} & \multicolumn{16}{|c|}{Maksymalny numer operacji (do sprawdzenia kompatybilności)} \\[3ex]
\hline
\textbf{12} & \textbf{96} & \multicolumn{16}{|c|}{Liczba etykiet w kodzie modułu}\\[3ex]
\hline
\textbf{16} & \textbf{128} & \multicolumn{16}{|c|}{Liczba funkcji eksportowanych z modułu} \\[3ex]
\hline
\textbf{20} & \textbf{160} & \multicolumn{8}{|c|}{Opkod 1} & \multicolumn{8}{|c|}{Argument 1}  \\[3ex]
\hline
\textbf{...} & \textbf{...} & \multicolumn{8}{|c|}{...} & \multicolumn{8}{|c|}{Argument N}  \\[3ex]
\hline
\textbf{...} & \textbf{...} & \multicolumn{8}{|c|}{Opkod 2} & \multicolumn{8}{|c|}{Argument 1}  \\[3ex]
\hline
\textbf{...} & \textbf{...} & \multicolumn{16}{|c|}{...}  \\[3ex]
\hline
\caption{Struktura kodu pośredniego w pliku BEAM}
\label{table:bytecode} \\
\end{longtable}


%---------------------------------------------------------------------------
\subsection{Tablica importowanych funkcji}
%---------------------------------------------------------------------------
Fragment pliku binarnego z tablicą importowanych funkcji zawiera informacje o funkcjach zaimplementowanych w innych modułach, które są wykorzystywane przez moduł.

Identyfikowany jest on przez napis \texttt{ImpT}. Struktura danych fragmentu zawarta została w tabeli \ref{table:importtable}.

\begin{longtable}{|c|c|c|c|c|c|c|c|c|c|c|c|c|c|c|c|c|c|}
\hline
         & \textbf{Oktet} & \multicolumn{8}{|c|}{\textbf{0}} & \multicolumn{8}{|c|}{\textbf{1}} \\
\hline
\textbf{Oktet} & \textbf{Bit} & \textbf{0} & \textbf{1} & \textbf{2} & \textbf{3} & \textbf{4} & \textbf{5} & \textbf{6} & \textbf{7} & \textbf{8} & \textbf{9} & \textbf{10} & \textbf{11} & \textbf{12} & \textbf{13} & \textbf{14} & \textbf{15}\\
\hline
\textbf{0} & \textbf{0} & \multicolumn{16}{|c|}{Liczba importowanych funkcji} \\[3ex]
\hline
\textbf{4} & \textbf{32} & \multicolumn{16}{|c|}{Indeks atomu z nazwą modułu 1}\\[3ex]
\hline
\textbf{8} & \textbf{64} & \multicolumn{16}{|c|}{Indeks atomu z nazwą funkcji 1} \\[3ex]
\hline
\textbf{12} & \textbf{96} & \multicolumn{16}{|c|}{Arność funkcji 1}\\[3ex]
\hline
\textbf{16} & \textbf{128} & \multicolumn{16}{|c|}{Indeks atomu z nazwą modułu 2}\\[3ex]
\hline
\textbf{...} & \textbf{...} & \multicolumn{16}{|c|}{...}  \\[3ex]
\hline
\caption{Struktura tablicy importowanych funkcji w pliku BEAM}
\label{table:importtable} \\
\end{longtable}

%---------------------------------------------------------------------------
\subsection{Tablica eksportowanych funkcji}
%---------------------------------------------------------------------------
Fragment pliku binarnego z tablicą eksportowanych funkcji zawiera informacje o funkcjach z modułu, które widoczne są z~poziomu innych modułów.

Identyfikowany jest on przez napis \texttt{ExpT}. Struktura danych fragmentu zawarta została w tabeli \ref{table:exporttable}.

\begin{longtable}{|c|c|c|c|c|c|c|c|c|c|c|c|c|c|c|c|c|c|}
\hline
         & \textbf{Oktet} & \multicolumn{8}{|c|}{\textbf{0}} & \multicolumn{8}{|c|}{\textbf{1}} \\
\hline
\textbf{Oktet} & \textbf{Bit} & \textbf{0} & \textbf{1} & \textbf{2} & \textbf{3} & \textbf{4} & \textbf{5} & \textbf{6} & \textbf{7} & \textbf{8} & \textbf{9} & \textbf{10} & \textbf{11} & \textbf{12} & \textbf{13} & \textbf{14} & \textbf{15}\\
\hline
\textbf{0} & \textbf{0} & \multicolumn{16}{|c|}{Liczba eksportowanych funkcji} \\[3ex]
\hline
\textbf{4} & \textbf{32} & \multicolumn{16}{|c|}{Indeks atomu z nazwą funkcji 1}\\[3ex]
\hline
\textbf{8} & \textbf{64} & \multicolumn{16}{|c|}{Arność funkcji 1} \\[3ex]
\hline
\textbf{12} & \textbf{96} & \multicolumn{16}{|c|}{Etykieta początku kodu funkcji 1}\\[3ex]
\hline
\textbf{16} & \textbf{128} & \multicolumn{16}{|c|}{Indeks atomu z nazwą funkcji 2}\\[3ex]
\hline
\textbf{...} & \textbf{...} & \multicolumn{16}{|c|}{...}  \\[3ex]
\hline
\caption{Struktura tablicy eksportowanych funkcji w pliku BEAM}
\label{table:exporttable} \\
\end{longtable}

%---------------------------------------------------------------------------
\subsection{Tablica funkcji lokalnych}
%---------------------------------------------------------------------------
Fragment pliku binarnego z tablicą lokalnych funkcji zawiera informacje o funkcjach zaimplementowanych w module (w tym lambd), które wykorzystywane są tylko przez ten moduł i nie są widoczne z~poziomu innych modułów.

Identyfikowany jest on przez napis \texttt{LocT}. Struktura danych fragmentu zawarta została w tabeli \ref{table:localtable}.

\begin{longtable}{|c|c|c|c|c|c|c|c|c|c|c|c|c|c|c|c|c|c|}
\hline
         & \textbf{Oktet} & \multicolumn{8}{|c|}{\textbf{0}} & \multicolumn{8}{|c|}{\textbf{1}} \\
\hline
\textbf{Oktet} & \textbf{Bit} & \textbf{0} & \textbf{1} & \textbf{2} & \textbf{3} & \textbf{4} & \textbf{5} & \textbf{6} & \textbf{7} & \textbf{8} & \textbf{9} & \textbf{10} & \textbf{11} & \textbf{12} & \textbf{13} & \textbf{14} & \textbf{15}\\
\hline
\textbf{0} & \textbf{0} & \multicolumn{16}{|c|}{Liczba lokalnych funkcji} \\[3ex]
\hline
\textbf{4} & \textbf{32} & \multicolumn{16}{|c|}{Indeks atomu z nazwą funkcji 1}\\[3ex]
\hline
\textbf{8} & \textbf{64} & \multicolumn{16}{|c|}{Arność funkcji 1} \\[3ex]
\hline
\textbf{12} & \textbf{96} & \multicolumn{16}{|c|}{Etykieta początku kodu funkcji 1}\\[3ex]
\hline
\textbf{16} & \textbf{128} & \multicolumn{16}{|c|}{Indeks atomu z nazwą funkcji 2}\\[3ex]
\hline
\textbf{...} & \textbf{...} & \multicolumn{16}{|c|}{...}  \\[3ex]
\hline
\caption{Struktura tablicy lokalnych funkcji w pliku BEAM}
\label{table:localtable} \\
\end{longtable}

%---------------------------------------------------------------------------
\subsection{Tablica lambd}
%---------------------------------------------------------------------------
Fragment pliku binarnego z tablicą lambd zawiera informacje o obiektach funkcyjnych, które wykorzystywane są przez ten moduł.

Lambdy indentyfikowane są poprzez atomy, które powstały przez złączenie nazwy funkcji, w~której zostały zdefiniowane oraz kolejny indeks lambdy zdefiniowanej w danej funkcji.
Np. kolejne obiekty funkcyjne zdefiniowane w funkcji \texttt{foo/1} będą identyfikowane przez atomy \texttt{-foo/1-fun-0-}, \texttt{-foo/1-fun-1-} itd.

Fragment pliku tablicą lambdy identyfikowany jest przez napis \texttt{FunT}. Struktura danych fragmentu zawarta została w tabeli \ref{table:lambdatable}.

\begin{longtable}{|c|c|c|c|c|c|c|c|c|c|c|c|c|c|c|c|c|c|}
\hline
         & \textbf{Oktet} & \multicolumn{8}{|c|}{\textbf{0}} & \multicolumn{8}{|c|}{\textbf{1}} \\
\hline
\textbf{Oktet} & \textbf{Bit} & \textbf{0} & \textbf{1} & \textbf{2} & \textbf{3} & \textbf{4} & \textbf{5} & \textbf{6} & \textbf{7} & \textbf{8} & \textbf{9} & \textbf{10} & \textbf{11} & \textbf{12} & \textbf{13} & \textbf{14} & \textbf{15}\\
\hline
\textbf{0} & \textbf{0} & \multicolumn{16}{|c|}{Liczba lambd w module} \\[3ex]
\hline
\textbf{4} & \textbf{32} & \multicolumn{16}{|c|}{Indeks atomu z identyfikatorem lambdy 1}\\[3ex]
\hline
\textbf{8} & \textbf{64} & \multicolumn{16}{|c|}{Arność lambdy 1} \\[3ex]
\hline
\textbf{12} & \textbf{96} & \multicolumn{16}{|c|}{Etykieta początku kodu lambdy 1}\\[3ex]
\hline
\textbf{16} & \textbf{128} & \multicolumn{16}{|c|}{Indeks lambdy 1 (0x00)}\\[3ex]
\hline
\textbf{20} & \textbf{160} & \multicolumn{16}{|c|}{Liczba wolnych zmiennych w lambdzie 1}\\[3ex]
\hline
\textbf{24} & \textbf{192} & \multicolumn{16}{|c|}{Wartość skrótu z drzewa syntaktycznego kodu lambdy 1}\\[3ex]
\hline
\textbf{28} & \textbf{224} & \multicolumn{16}{|c|}{Indeks atomu z identyfikatorem lambdy 2}\\[3ex]
\hline
\textbf{...} & \textbf{...} & \multicolumn{16}{|c|}{...}  \\[3ex]
\hline
\caption{Struktura tablicy lambd w pliku BEAM}
\label{table:lambdatable} \\
\end{longtable}

%---------------------------------------------------------------------------
\subsection{Tablica stałych}
%---------------------------------------------------------------------------
Fragment pliku binarnego z tablicą lambd stałych zawiera informacje o stałych (listy, napisy, duże liczby) które wykorzystywane są przez ten moduł.

Właściwa lista wartości stałych (od bajtu 4 do końca fragmentu) przechowywana jest w pliku w~postaci skompresowanej algorytmem \textbf{zlib}. Podany rozmiar w bajtach dotyczy nieskompresowanej tablicy stałych.
Stałe zapisane są w formacie binarnym w formacie \emph{External Term Format}, opisanym w~dokumencie \cite{ExternalTermFormat}.

Fragment pliku z tablicą identyfikowany jest przez napis \texttt{LitT}. Struktura danych fragmentu (w~zdekompresowanej postaci) zawarta została w tabeli \ref{table:literaltable}.

\begin{longtable}{|c|c|c|c|c|c|c|c|c|c|c|c|c|c|c|c|c|c|}
\hline
         & \textbf{Oktet} & \multicolumn{8}{|c|}{\textbf{0}} & \multicolumn{8}{|c|}{\textbf{1}} \\
\hline
\textbf{Oktet} & \textbf{Bit} & \textbf{0} & \textbf{1} & \textbf{2} & \textbf{3} & \textbf{4} & \textbf{5} & \textbf{6} & \textbf{7} & \textbf{8} & \textbf{9} & \textbf{10} & \textbf{11} & \textbf{12} & \textbf{13} & \textbf{14} & \textbf{15}\\
\hline
\textbf{0} & \textbf{0} & \multicolumn{16}{|c|}{Rozmiar tablicy w bajtach} \\[3ex]
\hline
\textbf{4} & \textbf{32} & \multicolumn{16}{|c|}{Liczba stałych}\\[3ex]
\hline
\textbf{8} & \textbf{64} & \multicolumn{16}{|c|}{Rozmiar stałej 1 w bajtach} \\[3ex]
\hline
\textbf{12} & \textbf{96} & \multicolumn{16}{|c|}{Stała 1 w External Term Format}\\[8ex]
\hline
\textbf{...} & \textbf{...} & \multicolumn{16}{|c|}{Rozmiar stałej 2 w bajtach}\\[3ex]
\hline
\textbf{...} & \textbf{...} & \multicolumn{16}{|c|}{...}\\[8ex]
\hline
\caption{Struktura tablicy stałych w pliku BEAM}
\label{table:literaltable} \\
\end{longtable}

%---------------------------------------------------------------------------
\subsection{Lista atrybutów modułu}
%---------------------------------------------------------------------------
Fragment pliku binarnego z listą atrybutów modułu zawiera listę dwójek (tzw. \emph{proplistę}) ze wszystkimi dodatkowymi atrybutami, z jakimi został skompilowany dany moduł (np. informacje o wersji czy autorze). Lista ta zapisana jest binarnie w postaci \emph{External Term Format}.

Fragment ten reprezentowany jest przez napis \texttt{Attr}.

%---------------------------------------------------------------------------
\subsection{Lista dodatkowych informacji o kompilacji modułu}
%---------------------------------------------------------------------------
Fragment pliku binarnego z listą informacji o kompilacji modułu zawiera proplistę z informacjami dotyczącymi kompilacji, takimi jak: ścieżka pliku z kodem źrodłowym, czas kompilacji, wersja kompilatora czy użyte opcje kompilacji. Informacje te zapisane są binarnie w postaci \emph{External Term Format}.

Fragment ten reprezentowany jest przez napis \texttt{CInf}.

%---------------------------------------------------------------------------
\subsection{Tablica linii kodu źródłowego modułu}
%---------------------------------------------------------------------------
Fragment pliku binarnego z informacjami o liniach kodu źródłowego modułu zawiera informacje dla instrukcji \texttt{line/1} maszyny wirtualnej o pliku źródłowym i linii, z której pochodzi aktualnie wykonywany fragment kodu. Informacje te wykorzystywane są przy generowaniu stosu wywołań przy wystąpięniu błędu lub wyjątku. Funkcjonalność ta została wprowadzona dopiero w wersji R15 maszyny wirtualnej BEAM.

Jeżeli kompilowany plik jest na etapie preprocessingu łączony z innymi plikami z kodem źródłowym (poprzez użycie atrybutu \texttt{include}) to informacja o tych plikach zostanie zawarta w tym fragmencie. Domyślnie, kompilowany plik nie zostanie uwzględniony i zostanie przydzielony mu indeks 0. 

Numer linii koduje się przy użyciu tagu \texttt{0001}, jak w przypadku argumentów instrukcji maszyny wirtualnej, opisanych w sekcji \ref{sec:opsTypes}.
Rozróżnienie pliku, z którego pochodzi linia odbywa się za pomocą zapamiętania, z którego pliku pochodziła ostatnia linia. Domyślnie jest to plik o indeksie 0. Jeżeli dochodzi do zmiany aktualnego pliku, kolejny numer linii poprzedzony jest indeksem pliku z którego pochodzi, zakodowanym przy użyciu tagu \texttt{0010} (jak w sekcji \ref{sec:opsTypes}). Dlatego też numer linii może zawierać w pliku binarnym 1 lub 2 bajty.

Fragment pliku z tablicą identyfikowany jest przez napis \texttt{Line}. Struktura danych fragmentu zawarta została w tabeli \ref{table:linetable}.

\begin{longtable}{|c|c|c|c|c|c|c|c|c|c|c|c|c|c|c|c|c|c|}
\hline
         & \textbf{Oktet} & \multicolumn{8}{|c|}{\textbf{0}} & \multicolumn{8}{|c|}{\textbf{1}} \\
\hline
\textbf{Oktet} & \textbf{Bit} & \textbf{0} & \textbf{1} & \textbf{2} & \textbf{3} & \textbf{4} & \textbf{5} & \textbf{6} & \textbf{7} & \textbf{8} & \textbf{9} & \textbf{10} & \textbf{11} & \textbf{12} & \textbf{13} & \textbf{14} & \textbf{15}\\
\hline
\textbf{0} & \textbf{0} & \multicolumn{16}{|c|}{Wersja (0x000000)} \\[3ex]
\hline
\textbf{4} & \textbf{32} & \multicolumn{16}{|c|}{Flagi (0x000000)}\\[3ex]
\hline
\textbf{8} & \textbf{64} & \multicolumn{16}{|c|}{Liczba instrukcji \texttt{line} w kodzie modułu} \\[3ex]
\hline
\textbf{12} & \textbf{96} & \multicolumn{16}{|c|}{Liczba linii z kodem w plikach modułu}\\[3ex]
\hline
\textbf{16} & \textbf{128} & \multicolumn{16}{|c|}{Liczba plików z kodem modułu}\\[3ex]
\hline
\textbf{20} & \textbf{160} & \multicolumn{8}{|c|}{Numer linii (1 lub 2 B)} & \multicolumn{8}{|c|}{Numer linii (1 lub 2 B)} \\[3ex]
\hline
\textbf{...} & \textbf{...} & \multicolumn{16}{|c|}{...}\\[3ex]
\hline
\textbf{...} & \textbf{...} & \multicolumn{8}{|c|}{Długość nazwy pliku 1} & \multicolumn{8}{|c|}{Nazwa pliku 1 w ASCII}\\[3ex]
\hline
\textbf{...} & \textbf{...} & \multicolumn{16}{|c|}{...}\\[3ex]
\hline
\textbf{...} & \textbf{...} & \multicolumn{8}{|c|}{Długość nazwy pliku 2} & \multicolumn{8}{|c|}{Nazwa pliku 2 w ASCII}\\[3ex]
\hline
\textbf{...} & \textbf{...} & \multicolumn{16}{|c|}{...}\\[3ex]
\hline
\caption{Struktura tablicy linii kodu źródłowego w pliku BEAM}
\label{table:linetable} \\
\end{longtable}

%---------------------------------------------------------------------------
\subsection{Drzewo syntaktyczne modułu}
%---------------------------------------------------------------------------
Plik z modułem zawiera fragment pliku źródłowego z drzewem syntaktycznym pliku z kodem źródłowym o ile został skompilowany z opcją \texttt{debug\_info}.
Fragment ten identyfikowany jest przez napis \texttt{Abst}. 

Zawartością fragmentu jest drzewo syntaktyczne modułu, w postaci opisanej w sekcji \ref{sec:compilationSyntaxtree} zakodowane w formacie \emph{External Term Format}.

\chapter{Lista instrukcji maszyny wirtualnej BEAM}
\label{cha:operacjeBeam}
%---------------------------------------------------------------------------

Dodatek zawiera listę instrukcji maszyny wirtualnej BEAM, jakie może zawierać skompilowany kod pośredni przez nią wykonywany.
Lista zawiera nazwę operacji, jej argumenty oraz opis jej działania.

Kod operacji zajmuje zawsze 1 bajt w pliku ze skompilowanym kodem pośrednim modułu.

Argumenty mogą zajmować więcej, zgodnie z opisem w sekcji \ref{sec:opsTypes}.

Kolejość bajtów w zapisie kodu pośredniego to \emph{big endian}.


\section{Typy argumentów}
\label{sec:opsTypes}
%---------------------------------------------------------------------------

Każdy z tagów jest możliwy do zapisania przy użyciu 3 bitów.
Jednak w kodowaniu binarnym do zapisu typu używane są dodatkowo 1 lub 2 bity. Dzięki nim możliwe jest rozróżnienie pomiędzy argumentami zapisanymi przy użyciu różnej liczby bajtów.

\begin{longtable}{|c|c|p{9cm}|}
\hline

\multicolumn{2}{|c|}{\textbf{Tag}} & \multirow{2}{*}{\textbf{Typ}} \\
\cline{1-2}
\textbf{binarnie} & \textbf{dziesiętnie} & \\
\hline
\endfirsthead

000 & 0 & uniwersalny indeks, np. do tablicy stałych \\
\hline
001 & 1 & liczba całkowita \\ 
\hline
010 & 2 & indeks do tablicy atomów \\
\hline
011 & 3 & numer rejestru X maszyny wirtualnej \\
\hline
100 & 4 & numer rejestru Y maszyny wirtualnej \\
\hline
101 & 5 & etykieta, używana w funkcjach skoku \\
\hline
111 & 7 & złożone wyrażenie (np. lista, liczba zmiennoprzecinkowa) \\
\hline

\caption{Tagi typów danych w pliku ze skompilowanym modułem}  \\
\end{longtable}

Jeżeli tagowana liczba jest nieujemna, mniejsza od 16 (możliwe jest zapisanie jej przy użyciu 4 bitów) to argument jest zapisany przy użyciu jednego bajtu a jego postać binarna to:
$$ \text{X}_1\text{X}_2\text{X}_3\text{X}_4\mathbf{0}\text{\textbf{TTT}}_{(2)}, $$
gdzie ${\text{X}_1\text{X}_2\text{X}_3\text{X}_4}_{(2)}$ to tagowana liczba, $\text{X}_1$ jest jej najbardziej znaczącym bitem, a $\text{TTT}_{(2)}$ to tag danego typu argumentu.

Na przykład, atom, który w tablicy atomów modułu ma indeks $2_{10} = 10_{2}$, po zakodowaniu będzie miał postać:
$$0010\mathbf{0010}_{2} = 22_{16} = 34_{10}.$$

W przypadku, gdy liczba jest nieujemna, mniejsza lub równa 16, a mniejsza od 2048 (możliwe jest jej zapisanie przy użyciu 11 bitów), argument jest zapisany przy użyciu dwóch bajtów, których postać binarna to:
$$  {\text{X}_1\text{X}_2\text{X}_3\mathbf{01}\text{\textbf{TTT}} \enskip \text{X}_4\text{X}_5\text{X}_6\text{X}_7\text{X}_8\text{X}_9\text{X}_{10}\text{X}_{11}}_{(2)}, $$
gdzie ${\text{X}_1 ... \text{X}_{11}}_{(2)}$ to tagowana liczba, $\text{X}_1$ jest jej najbardziej znaczącym bitem, a ${\text{TTT}}_{(2)}$ to tag danego typu argumentu.

Na przykład, liczba całkowita $565_{10} = {010 \enskip 00110101}_{2}$ po zakodowaniu będzie miała postać:
$${010\mathbf{01001} \enskip 00110101}_{2} = 4935_{16} = 18741_{10}.$$

Jeżeli argument jest liczbą ujemną lub dodatnią wymagającą w zapisie dwójkowym więcej niż 11 bitów to liczba taka zapisywana jest binarnie w kodzie uzupełnień do dwóch poprzedzona odpowiednim nagłówkiem.

Jeżeli zakodowaną liczbę można zapisać na nie więcej niż 8 bajtach, to nagłówek ma następującą postać:

$$ {\text{N}_1\text{N}_2\text{N}_3 \mathbf{11} \text{\textbf{TTT}}}_{(2)}, $$
gdzie ${\text{N}_1\text{N}_2\text{N}_3}_{(2)}$ to rozmiar argumentu w bajtach pomniejszony o 2 (jeżeli argument jest liczbą ujemną zajmującą 1 bajt to powinien on zostać dopełniony do 2 bajtów), $\text{N}_1$ jest jego najbardziej znaczącym bitem, a $\text{TTT}_{(2)}$ to tag danego typu argumentu.

Na przykład, aby zapisać na dwóch bajtach liczbę $-21_{10} = {11111111 \enskip 11101011}_{U2}$, jej postać binarną należy poprzedzić nagłówkiem:

$${000\mathbf{11001}}_{2} = 19_{16} = 25_{10}.$$

Jeżeli do zapisania liczby w kodzie uzupełnień do dwóch potrzeba przynajmniej 9 bajtów, wtedy nagłówek ma postać:

$$ {11111\text{\textbf{TTT}} \enskip \text{N}_1\text{N}_2\text{N}_3\text{N}_4 \mathbf{0000}}_{(2)}, $$
gdzie ${\text{N}_1\text{N}_2\text{N}_3\text{N}_4}_{(2)}$ to rozmiar argumentu w bajtach pomniejszony o 9, $\text{N}_1$ jest jego najbardziej znaczącym bitem, a $\text{TTT}_{(2)}$ to tag danego typu argumentu.

Na przykład, w celu zapisania liczby $2^{(15 \times 8)-1}-1$ na 15 bajtach, należy zapis tej liczby w kodzie U2 poprzedzić następującym nagłówkiem:

$${11111\mathbf{001} \enskip 0110\mathbf{0000}}_{2} = \text{F}960_{16} = 63840_{10}.$$

\section{Lista instrukcji}
\label{sec:opsOps}
%---------------------------------------------------------------------------

\begin{longtable}{|c|c|c|p{5cm}|}
\hline

\multicolumn{2}{|c|}{\textbf{Kod operacji}} & \multirow{2}{*}{\textbf{Nazwa operacji i jej argumenty}} & \multirow{2}{*}{\textbf{Opis operacji i uwagi}} \\
\cline{1-2}
\textbf{szesnastkowo} & \textbf{dziesiętnie} & & \\
\hline
\endfirsthead

01 & 1 & \texttt{label Lbl} & Wprowadza lokalną dla danego modułu etykietę identyfikującą aktualne miejsce w kodzie. \\
\hline
02 & 2 & \texttt{func\_info M F A} & Definiuje funkcję \texttt{F}, w module \texttt{M} o arności \texttt{A}. \\
\hline
03 & 3 & \texttt{int\_code\_end} & ???  \\
\hline

\caption{Lista operacji maszyny wirtualnej BEAM}  \\
\end{longtable}





















\bibliographystyle{plain}
\addcontentsline{toc}{chapter}{Bibliografia}
\bibliography{bibliografia}
%\begin{thebibliography}{1}
%
%\bibitem{Dil00}
%A.~Diller.
%\newblock {\em LaTeX wiersz po wierszu}.
%\newblock Wydawnictwo Helion, Gliwice, 2000.
%
%\bibitem{Lam92}
%L.~Lamport.
%\newblock {\em LaTeX system przygotowywania dokumentów}.
%\newblock Wydawnictwo Ariel, Krakow, 1992.
%
%\bibitem{Alvis2011}
%M.~Szpyrka.
%\newblock {\em {On Line Alvis Manual}}.
%\newblock AGH University of Science and Technology, 2011.cccccc
%\newblock \\\texttt{http://fm.ia.agh.edu.pl/alvis:manual}.
%
%\end{thebibliography}

\end{document}
