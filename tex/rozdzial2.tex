\chapter{System operacyjny FreeRTOS}
\label{cha:freertos}

%---------------------------------------------------------------------------
\section{Wprowadzenie}
\label{sec:rtosWprowadzenie}

Podstawową częścią każdego systemu operacyjnego jest jego jądro
FreeRTOS jest mikrojądrem, umożliwiającym tworzenie aplikacji czasu rzeczywistego (zarówno o miękkich jak i twardych wymaganiach) przeznaczonych na systemy wbudowane. 

%---------------------------------------------------------------------------
\section{Zadania i planista (\emph{scheduler})}
\label{sec:rtosScheduler}

%---------------------------------------------------------------------------
\section{Kolejki}
\label{sec:rtosKolejki}

%---------------------------------------------------------------------------
\section{Przerwania}
\label{sec:rtosPrzerwania}

%---------------------------------------------------------------------------
\section{Zarządzanie zasobami}
\label{sec:rtosZasoby}

%---------------------------------------------------------------------------
\section{Zarządzanie pamięcią}
\label{sec:rtosPamiec}

%---------------------------------------------------------------------------
\section{FreeRTOS i LPC17xx}
\label{sec:rtosLPC}

Mikrokontroler LPC17xx jest jednym z systemów, na który przeniesiony został mikrojądrem FreeRTOS.

%---------------------------------------------------------------------------
\section{Podsumowanie}
\label{sec:rtosPodsumowanie}