\chapter{System operacyjny FreeRTOS}
\label{cha:freertos}

\textbf{Dodaj odnośniki do opisanych funkcjonalności maszyny, gdy ogólnie wspomniane}

Podstawową częścią każdego systemu operacyjnego jest jego jądro, które odpowiedzialne jest za udostępnianie zasobów sprzętowych, takich jak procesor, pamięć, czy urządzenia wejścia/wyjścia, programom wykonywanym na tym systemie.

System FreeRTOS jest mikrojądrem (por. architektury oprogramowania na str. \pageref{ref:architektury}), przy użyciu którego możliwa jest implementacja aplikacji czasu rzeczywistego (zarówno o miękkich jak i twardych wymaganiach) na urządzeniach wbudowanych.

W niniejszym rozdziale opisano architekturę systemu (mikrojądra) FreeRTOS wraz ze sposobem, w jaki
poszczególne funkcjonalności mogą być przydatne w implementacji maszyny wirtualnej Erlanga dedykowanej dla tego systemu.

%---------------------------------------------------------------------------
\section{Zadania i planista (\emph{scheduler})}
\label{sec:rtosScheduler}

Podstawową wykonywalną jednostką w systemie FreeRTOS jest zadanie, zarządzane przez wbudowanego w system planistę (\emph{scheduler}).
Zadanie uruchomione pod nadzorem planisty można porównać do wątku w systemie Linux, z tą różnicą, że kod zadania musi zostać zaimplementowany w języku C i przed rozpoczęciem jego wykonywania należy zadeklarować rozmiar stosu danego zadania.

\emph{Scheduler} może pracować w dwóch trybach: wywłaszczeniowym, w którym sam algorytm planisty decyduje o kolejności wykonywania zadań, oraz w trybie opartym na współpracy, w którym zadania ,,dobrowolnie'' rezygnują z czasu procesora, który został im przydzielony. W tym drugim przypadku, priorytety zadań są nadal brane pod uwagę podczas wyboru kolejnego zadania do wykonania.

Wielozadaniowość oparta na współpracy to model, jaki został zaimplementowany w oryginalnej maszynie wirtualnej Erlanga, po wprowadzeniu pojęcia redukcji jako miary czasu, przez jaki danemu procesowi udostępniona jest moc obliczeniowa (por. rozdział \ref{cha:erlang}).

Wymienione cechy charakterystyczne zadań i planisty stanowią dobry punkt wyjścia do oparcia implementacji \emph{schedulera} maszyny wirtualnej Erlanga w oparciu o planistę systemu FreeRTOS oraz enkapsulację logiki procesów w zadaniach.

%---------------------------------------------------------------------------
\section{Kolejki}
\label{sec:rtosKolejki}

System FreeRTOS zapewnia mechanizm kolejki wiadomości między procesami, na wzór kolejki wiadomości POSIX. Kolejki nie należą do żadnego z zadań, dlatego też każde z zadań może zarówno odczytywać jak i zapisywać danego do każdej kolejki. Proces przesłania i odebrania wiadomości polega na skopiowaniu danych z przestrzeni adresowej nadawcy do przestrzeni adresowej kolejki a następnie z przestrzeni adresowej kolejki do przestrzeni adresowej zadania adresata.

Kolejki w systemie FreeRTOS bardzo dobrze oddają semantykę kolejki wiadomości (\emph{mailbox}) w procesie Erlangowym. Jednakże istniałaby konieczność utworzenia osobnej kolejki dla każdego z uruchomionych w systemie procesów, do czego konieczne jest zaalokowanie pamięci dla kolejki wiadomości o maksymalnej długości z góry. 

W związku z tym, w maszynie wirtualnej opisanej w niniejszej pracy kolejki wiadomości zostały zaimplementowane wewnątrz zadań implementujących logikę procesów. Pozwoli to na uproszczenie procedury wysłania wiadomości do procedury przez umieszczenie wiadomości na stercie procesu będącego jej adresatem, z której proces będzie mógł korzystać aż do momentu odśmiecenia pamięci procesu. 

%---------------------------------------------------------------------------
\section{Przerwania}
\label{sec:rtosPrzerwania}

FreeRTOS zapewnia obsługę zarówno programowych jak i sprzętowych przerwań. Podejściem implementacji obsługi przerwań przez autorów systemu jest ich odroczenie i delegacja obsługi do innego zadania, niż to które obsługuje przerwanie przez \emph{Interrupt Service Routine} (ISR) \cite{Barry2011}. Motywacją do tego, aby kod ISR był możliwie jak najkrótszy jest fakt, że w momencie jego wykonywania nowe przerwania nie są identyfikowane.

W implementacji maszyny wirtualnej Erlanga dla FreeRTOS podążono za tą koncepcją i informacja o przerwaniu jest przesyłana jako wiadomość do procesów, które wywołają wcześniej odpowiednią funkcję subskrybującą.

%---------------------------------------------------------------------------
\section{Zarządzanie zasobami}
\label{sec:rtosZasoby}

%---------------------------------------------------------------------------
\section{Zarządzanie pamięcią}
\label{sec:rtosPamiec}

%---------------------------------------------------------------------------
\section{FreeRTOS i LPC17xx}
\label{sec:rtosLPC}

Mikrokontroler LPC17xx jest jednym z systemów, na który przeniesiony został mikrojądrem FreeRTOS.

%---------------------------------------------------------------------------
\section{Podsumowanie}
\label{sec:rtosPodsumowanie}