\chapter{Język programowania Erlang}
\label{cha:erlang}

Niniejszy rozdział przybliża podstawy języka programowania Erlang.
Zaprezentowany podzbiór języka został zaprezentowany w takim zakresie, aby możliwe było zrozumienie zadań, jakie do zrealizowania mają poszczególne elementy maszyny wirtualnej, zaimplementowane w niniejszej pracy.
Szczegółowy opis języka został zawarty np. w pozycjach \cite{Armstrong2013} i \cite{Hebert2013}.


Erlang jest wieloparadygmatowym --- funkcyjnym i współbieżnym językiem programowania ogólnego przeznaczenia o dynamicznym, lecz silnym typowaniu i automatycznym systemie zarządzania pamięcią.
Kod źródłowy w Erlangu kompilowany jest do kodu pośredniego, który następnie interpretowany jest przez dedykowaną maszynę wirtualną.
Motywacje, które przyczyniły się do zaprojektowania i~implementacji języka oraz jego podstawowe założenia zostały przedstawione w podrozdziale \ref{sec:jezykiFunkcyjne}.

%---------------------------------------------------------------------------
\section{Typy danych}
\label{sec:erlangTypy}

Język definiuje osiem podstawowych typów danych:
\begin{itemize}
\item liczby całkowite --- operacje arytmetyczne na danych tego typu zapewnione są z nieograniczoną precyzją (ograniczoną tylko przez dostępną pamięć), dzięki wprowadzeniu do maszyny wirtualnej własnej arytmetyki stałoprzecinkowej. Przykładowymi wyrażeniami tego typu są \texttt{10} i \texttt{-25};
\item atomy --- wyrażenia identyfikowane przez ciągi znaków zaczynające się małą literą lub zawarte w~pojedynczych cudzysłowach, w maszynie wirtualnej są jednak zamieniane na liczbę całkowitą w~celu szybszego porównywania ich. Przykładowymi atomami są \texttt{erlang} i \texttt{'EXIT'};
\item liczby zmiennoprzecinkowe --- typ danych reprezentuje liczby rzeczywiste z 64-bitową precyzją, np. \texttt{3.14} czy \texttt{-2.718};
\item referencje --- typ danych reprezentujący unikalne wyrażenie w zakresie klastra maszyn połączonych w sieć, służące do identyfikacji innych wyrazeń. Zmienna tego typu może zostać utworzona wyłącznie przez wywołanie funkcji \texttt{make\_ref/0} wbudowanej w maszynę wirtualną;
\item binaria --- to ciągi bajtów zajmujących ciągły obszar pamięci, np. \texttt{<<255,255,255,0>>}; 
\item identyfikatory procesów --- pozwalające na odniesienie się do wystartowanego procesu w maszynie wirtualnej poprzez wysłanie wiadomości lub zamknięcie go;
\item porty --- typ danych używany do komunikacji z systemem operacyjnym, np. systemem plików czy stosem sieciowym;
\item lambdy --- obiekty funkcyjne, które mogą zostać przekazane jako argument do funkcji wyższego rzędu i w niej wywołane.
\end{itemize}

Dodatkowo, zdefiniowane zostały dwa typy złożone:
\begin{itemize}
\item krotki --- przechowujące określoną z góry liczbę innych wyrażeń (prostych lub złożonych). Dostęp do dowolnego obiektu w krotce możliwy jest w czasie stałym. Przykładem krotki jest \texttt{\{salary, 100, 4.50}\};
\item listy --- przechowujące inne wyrażenia (proste lub złożone) na liście jednokierunkowej. Dostęp do dowolnego obiektu na liście możliwy jest w czasie liniowym. Przykładem listy jest \texttt{[salary, 100, 4.50]}.
\end{itemize}

Oprócz tego, język zapewnia dwa ,,lukry składniowe'', które na etapie kompilacji kodu źródłowego zamieniane są na wymienione wcześniej typy danych:
\begin{itemize}
\item napisy --- zapisywane jako ciąg znaków zawartych w podwójnych cudzysłowach, które zamieniane są na listę kodów ASCII poszczególnych znaków. Np. napis \texttt{"hello"} jest tak naprawdę listą postaci \texttt{[104,101,108,108,111]};
\item rekordy --- pozwalające na odnoszenie się do poszczególnych pól krotki z użyciem nazwy (atomu), co upraszcza posługiwanie się tym typem danych.
\end{itemize}

%---------------------------------------------------------------------------
\section{Moduły i dynamiczna podmiana kodu}
\label{sec:erlangModuly}

Jednostką kompilacji kodu źródłowego w języku Erlang jest pojedynczy moduł (plik z rozszerzeniem \textbf{*.erl}), którego proces kompilacji opisany został w dodatku \ref{cha:erlangKompilacja}.
Wszystkie skompilowane moduły nie są zależne od żadnych innych, dlatego procesowi kompilacji nie towarzyszy linkowanie modułów.
Zależności pomiędzy modułami rozwiązywane są już w trakcie uruchomienia systemu, przez maszynę wirtualną. 

Pojedynczy moduł składa się z zestawu funkcji: lokalnych i zewnętrznych.
Funkcje lokalne możliwe są do użycia tylko i wyłącznie przez kod w danym module, natomiast funkcje zewnętrzne mogą być używane przez dowolny inny moduł.
Dana funkcja jest funkcją zewnętrzną jeżeli została jawnie wyeksportowana z danego modułu poprzez użycie dyrektywy kompilatora \texttt{-export}.

Poszczególne funkcje rozróżniane są na podstawie: modułu w którym zostały zdefiniowane, nazwy oraz arności (liczby przyjmowanych argumentów).
Na przykład, funkcja \texttt{bar} zdefiniowana w module \texttt{foo}, przyjmująca dwa argumenty oznaczana jest symbolem \texttt{foo:bar/2}.

Modularność implementowanych aplikacji pozwoliła na wprowadzenie do maszyny wirtualnej języka kolejnej ważnej cechy --- możliwości dynamicznej podmiany kodu.
Załadowanie nowej wersji danego modułu możliwe jest w każdej chwili uruchomienia maszyny wirtualnej.
Nie ma to jednak wpływu na wykonanie procesów korzystających ze starej wersji kodu, gdyż maszyna może przechowywać dwie różne.
W momencie, gdy tej wersji modułu nie będzie wykorzystywał już żaden proces zostanie on usunięty z pamięci.

%---------------------------------------------------------------------------
\section{Kontrola przebiegu programu}
\label{sec:erlangFlow}

Ponieważ Erlang jest funkcyjnym językiem programowania, podstawą sterowania programem jest wywoływanie funkcji lokalnych, zdefiniowanych w tym samym module, z którego następuje wywołanie lub zewnętrznych, zdefiniowanych w innych modułach lub funkcji wbudowanych, zaimplementowanych bezpośrednio w kodzie maszyny wirtualnej.

W treści funkcji wykorzystywać można zmienne.
Zmienne w języku Erlang reprezentowane są za pomocą napisów rozpoczynających się od wielkiej litery.
Zmienna może znajdować się w dwóch stanach: związanym z pewnością wartością lub niezwiązanym.
Jeżeli pewnej zmiennej przypisano jakąś wartość nie można jej już zmienić w tej samej funkcji.
Próba przypisania zmiennej nowej wartości zakończy się błędem \texttt{badmatch}, natomiast próba ponownego powiązania zmiennej z już przypisaną wcześniej wartością jest dozwolona i stanowi naturalny sposób implementacji asercji w kodzie.
Początkowym zbiorem zmiennych z przypisaną wartością dla funkcji są jej argumenty.

Wybór odpowiedniego fragmentu kodu do wykonania w zależności od wartości zmiennych możliwy jest dzięki mechanizmowi \emph{pattern matchingu}.
Polega on na wyborze pierwszej gałęzi funkcji (w przypadku zastosowaniu w nagłówkach funkcji) lub instrukcji warunkowej (w przypadku instrukcji \texttt{case} i~\texttt{if}), która jest zgodna z przekazanymi wartości zmiennych.
Nagłówek funkcji lub gałąź instrukcji warunkowej może definiować dokładną wartość jaką powinny mieć zmienne (w tym wartości związanych już zmiennych) lub częściową (wiążąc nieznane wartości z nowymi zmiennymi).
Oprócz tego możliwe jest wykonywanie dodatkowych sprawdzeń spełnienia dodatkowych warunków przy pomocy zestawu strażników (\emph{guards}).

Przykład \emph{pattern matchingu} został zaprezentowany na listingu \ref{lis:patternmatching}.
Jednoargumentowa funkcja \texttt{foo} zwróci atom \texttt{a} jeżeli jej argumentem będzie liczba całkowita.
Atom \texttt{b} zostanie zwrócony w przypadku, gdy przekazanym argumentem będzie krotka o arności 2, której pierwszym elementem będzie atom \texttt{arg}. Drugi element krotki zostanie wtedy powiązany ze zmienną \texttt{Arg}. W sytuacji, gdy będzie to każda inna dwójka, z funkcji zostanie zwrócona wartość \texttt{c}. Pierwszy element krotki zostanie powiązany ze zmienną \texttt{Arg1}, zaś drugi ze zmienną \texttt{Arg2}.
Dla każdego innego przekazanego argumentu funkcja zwróci atom \texttt{d}.

\begin{lstlisting}[style=erlang, caption=Przykład \emph{pattern matchingu}, label=lis:patternmatching]
foo(Arg) when is_integer(Arg) ->
    a;
foo({arg, Arg}) ->
    b;
foo({Arg1, Arg2}) ->
    c;
foo(Arg) ->
    d.
\end{lstlisting}

Ze względu na to, że języki funkcyjne nie mają konstrukcji pętli znanych z języków imperatywnych, wszelkie powtarzalne operacje powinny być zaimplementowane przy użyciu funkcji rekurencyjnych.
Kompilator języka Erlang identyfikuje funkcje ogonowo-rekurencyjne, których zwracana wartość zależy tylko i wyłącznie od rekurencyjnego wywołania funkcji. Dzięki temu możliwa jest optymalizacja kodu pośredniego tak, aby do rekurencyjnych wywołań funkcji nie wykorzystywać stosu.

%---------------------------------------------------------------------------
\section{Programowanie współbieżne i rozproszone}
\label{sec:erlangConcurrent}

Erlang jest implementacją modelu aktorów \cite{Hewitt73}.
Pojedynczy proces wykonujący kod jest całkowicie odizolowany od innych procesów w systemie.
Jedynym sposobem komunikacji pomiędzy procesami jest wysłanie asynchronicznej wiadomości do innego procesu, która kopiowana jest do kolejki wiadomości w jego przestrzeni adresowej.
Brak pamięci współdzielonej pomiędzy procesami pozwala na całkowite wyeliminowanie z implementacji programu mechanizmów zarządzania zasobami, takich jak semafory. 

Wiadomości wysyłane są do innych procesów za pomocą operatora \texttt{!}.
Odbierane z kolei są w bloku \texttt{receive}, który na podstawie dopasowania zawartości wiadomości za pomocą \emph{pattern matchingu} wykonuje odpowiednią gałąź kodu.

Maszyna wirtualna Erlanga została zaprojektowana w ten sposób, aby na pojedynczej fizycznej maszynie możliwe było uruchomienie bardzo dużej (rzędu setek tysięcy) procesów. Było to możliwe dzięki własnej implementacji logiki procesów, nie zaś korzystanie z wymagających wątków systemu operacyjnego. Za kolejność wykonywania procesów odpowiada wewnętrzny planista (\emph{scheduler}).
Gdy na docelowej maszynie procesor może obsłużyć więcej fizycznych wątków, maszyna wirtualna uruchamia więcej instancji planisty, co prowadzi do automatycznego zrównoleglenia uruchamianej aplikacji.

Proces tworzony jest przez wywołanie funkcji wbudowanej z rodziny \texttt{spawn}, ze wskazaniem na to jaka funkcja lub obiekt funkcyjny jest początkowym punktem wykonywanego kodu procesu.
Wartością zwracaną przez tę funkcję jest identyfikator procesu, którego można użyć do wysłania wiadomości do wystartowanego procesu.
Możliwe jest także zarejestrowanie procesu w maszynie wirtualnej pod daną nazwą (atomem) za pomocą funkcji wbudowanej \texttt{register/2}, dzięki czemu procesy wysyłające wiadomość do procesu nie potrzebują znać jego identyfikatora.

Proces uruchomiony jest tak długo, jak wykonuje się uruchomiona funkcja. Dlatego powszechną praktyką jest wykonywanie w ramach procesu nieskończonej pętli, która odbiera wiadomości kierowane do procesu, na podstawie ich treści wykonuje odpowiednią akcję po czym wykonuje się od początku.

Programy napisane w Erlangu mogą być w łatwy sposób przeniesione do środowiska rozproszonego.
Z punktu widzenia programisty, wysłanie wiadomości do procesu uruchomionego na innym węźle w~tym samym klastrze, niczym się nie różni.
Możliwe jest również uruchomienie procesu na zdalnym węźle oraz monitorowanie jego stanu.
Klaster tworzony jest za pomocą połączeń TCP nawiązanych przez maszyny wirtualne uruchomione na fizycznych maszynach połączonych w sieć.

%---------------------------------------------------------------------------
\section{Obsługa błędów}
\label{sec:erlangBledy}

System obsługi błędów występujących w trakcie działania programu jest jedną z bardzo ważnych cech języka. 

Proces może przerwać swoje działanie i wygenerować wyjątek w następujących sytuacjach:
\begin{itemize}
\item do funkcji wbudowanej przekazana zostanie zmienna właściwego typu, ale o złej wartości (np. dzielenie przez zero);
\item do funkcji wbudowanej zostanie przekazana zmienna niewłaściwego typu;
\item próbie dopasowania w konstrukcji \emph{pattern matchingu} zostanie poddane nieprzewidziane wyrażenie;
\item z błędem zakończony zostanie proces połączony z danym;
\item wygenerowany zostanie wyjątek z poziomu systemu operacyjnego (np. brak wolnej pamięci);
\item wyjątek zostanie wygenerowany jawnie przez wywołanie w kodzie przez programistę jednej z~funkcji \texttt{exit/1}, \texttt{error/1} lub \texttt{throw/1}.
\end{itemize}

Występujące błędy mogą być obsługiwane na dwóch poziomach: wewnątrz procesów i pomiędzy procesami.

Międzyprocesowa obsługa błędów może wykorzystywać dwa mechanizmy: dwukierunkowych linków oraz jednokierunkowych monitorów.
Pierwszy z nich służy do propagacji informacji o błędzie wśród wszystkich połączonych ze sobą procesów.
Jeżeli jeden z połączonych procesów kończy swoje działanie na skutek błędu, każdy z połączonych z nim procesów również zakończy swoje działanie na skutek tego samego błędu itd.
Istnieje jednak możliwość przechwycenia informacji o błędzie w innym procesie, zamiast zakończenia swojego działnia, i na tej podstawie dokonania pewnej operacji.

Monitory z kolei zapewniają, że w sytuacji zakończenia działania procesu, proces monitorujący go otrzyma o tym fakcie informację w postaci odpowiedniej wiadomości.

Obsługa błędów wewnątrz procesu skutkuje brakiem jego zakończenia, wykonywany jest zaś odpowiedni fragment kodu w zależności od rodzaju błędu. Służą do tego dwie konstrukcje języka: \texttt{catch} oraz \texttt{try}.

Pierwsza z nich stara się znaleźć wartość wyrażenia wykonując kod, która jednak przybiera postać krotki \texttt{\{'EXIT', ...\}} w przypadku błędu.
Druga z konstrukcji pozwala na przeprowadzenie \emph{pattern matchingu} w zależności od rodzaju błędu, a także na wykonanie kodu który zostanie wykonany po wykonaniu całego kodu, niezależnie od tego czy w trakcie wystąpił błąd czy nie.

%---------------------------------------------------------------------------
\section{Podsumowanie}
\label{sec:erlangPodsumowanie}

Erlang jest funkcyjnym językiem programowania, który został zaprojektowany w celu łatwej implementacji współbieżnych aplikacji.
Kod źródłowy aplikacji podzielony jest na moduły, które mogą być ładowane dynamicznie w trakcie uruchomienia systemu.
Ponadto, wbudowane mechanizmy obsługi błędów pozwalają na łatwą implementację aplikacji odpornych na błędy.

Podstawowymi jednostkami programu uruchomionego wewnątrz maszyny wirtualnej są odizolowane od siebie procesy, które mogą się między sobą komunikować tylko i wyłącznie za pomocą wysyłanych do siebie, asynchronicznych wiadomości.
Język ułatwia tworzenie aplikacji rozproszonych wprowadzając przezroczystość lokalizacji procesów dla programisty, dla którego różnicy nie stanowi czy proces uruchomiony jest na lokalnym czy zdalnym węźle.
