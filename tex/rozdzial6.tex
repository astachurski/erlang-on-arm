\chapter{Podsumowanie}
\label{cha:podsumowanie}

W ramach niniejszej pracy dyplomowej dokonano implementacji maszyny wirtualnej języka programowania Erlang dla mikrojądra FreeRTOS, przeznaczonego do uruchamiania na urządzeniach wbudowanych, w podstawowym zakresie, pozwalającym na uruchamianie prostych aplikacji napisanych w tym języku.

Z wykorzystaniem zaimplementowanej maszyny napisano i uruchomiono na mikrokontrolerze LPC1769 trzy przykładowe aplikacje.
Ich działanie potwierdziło właściwe funkcjonowanie m.in. sekwencyjnych i współbieżnych cech języka, stabilności działania w czasie czy możliwość praktycznego wykorzystania do implementacji sterownika urządzenia peryferyjnego.

Ponadto, praca stanowi dokumentację sposobu działania poszczególnych elementów maszyny wirtualnej Erlanga, razem z zestawieniem z maszyną wirtualną BEAM, która jest najpopularniejszym środowiskiem uruchomieniowym dla języka, na której implementacja maszyny została oparta.

Udokumentowano również pośrednie postacie kodu źródłowego programu w języku Erlang jakie wykorzystuje kompilator tego języka, strukturę skompilowanego pliku z kodem pośrednim oraz znaczenie instrukcji, jakie można znaleźć w plikach tego typu.
Tematyka ta nie jest objęta przez oficjalną dokumentację języka, pewna jej postać jednak była niezbędna do wykonania implementacji maszyny. 

W ramach przyszłych prac warto dokonać próby rozwoju projektu w następujących aspektach:

\begin{itemize}
\item rozbudowa narzędzia budującego aplikacje, tak aby możliwe było zbudowanie maszyny wirtualnej ze skompilowanymi modułami wraz z odpowiednim portem systemu FreeRTOS w jednym kroku;
\item implementacja pozostałych opkodów;
\item implementacja pozostałych typów danych, takich jak binaria czy referencje;
\item implementacja własnej logiki planisty, co pozwoli na oszczędność pamięci używanej przez zadanie systemu FreeRTOS;
\item implementacja protokołu \emph{Distributed Erlang} \cite{DistributedErlang} i integracja go ze stosem TCP/IP dedykowanym dla systemu FreeRTOS;
\item umożliwienie ładowania kodu do pamięci za pomocą \emph{Distributed Erlang}, dzięki czemu wkompilowywanie kodu modułów w kod maszyny wirtualnej przestanie być konieczne.
\end{itemize}