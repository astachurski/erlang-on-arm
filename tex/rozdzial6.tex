\chapter{Podsumowanie}
\label{cha:podsumowanie}

\section{Wnioski}
\label{sec:podsumowanieWnioski}

\section{Dalszy rozwój projektu}
\label{sec:podsumowanieDalszy}

W ramach przyszłych prac warto dokonać próby rozwoju projektu w następujących aspektach:

\begin{itemize}
\item rozbudowa narzędzia budującego aplikacje, tak aby możliwe było zbudowanie maszyny wirtualnej ze skompilowanymi modułami wraz z odpowiednim portem systemu FreeRTOS w jednym kroku;
\item implementacja pozostałych opkodów;
\item implementacja pozostałych typów danych, takich jak binaria czy referencje;
\item implementacja własnej logiki planisty, co pozwoli na oszczędność pamięci używanej przez zadanie systemu FreeRTOS;
\item implementacja protokołu \emph{Distributed Erlang} \cite{DistributedErlang} i integracja go ze stosem TCP/IP dedykowanym dla systemu FreeRTOS;
\item umożliwienie ładowania kodu do pamięci za pomocą \emph{Distributed Erlang}, dzięki czemu wkompilowywanie kodu modułów w kod maszyny wirtualnej przestanie być konieczne.
\end{itemize}